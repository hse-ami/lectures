\documentclass[a4paper, 12pt]{article}
\usepackage{header}

\begin{document}
\newcommand{\B}{\mathcal{B}}
\newcommand{\D}{\mathcal{D}}

\pagestyle{fancy}
\section{Лекция 10 от 07.11.2016 \\ Критерий Гордона}
В прошлый раз мы узнали, что такое база множества и понятие предела по базе, и теперь будем продолжать работать с этим.
\par Рассмотрим такую задачу
\begin{Problem}
    Пусть $X$ и $Y$ --- непустые множества с базами $\B$ и $\D$ соответственно. Тогда Рассмотрим некоторую функцию $h\colon \;X\times Y \to \mathbb{R}$. Пусть про неё известно, что
    \begin{gather*}
        \forall x \in X\; \exists \lim\limits_{\D}h(x,y) = f(x) \\
        \forall y \in Y\; \exists \lim\limits_{\B} h(x,y) = g(y)
    \end{gather*}
    Требуется узнать, равны ли пределы $\lim\limits_{\B} f(x)$ и $\lim\limits_{\D} g(x)$. То есть верно ли, что
    $$
        \lim\limits_{\B}\lim\limits_{\D} h(x,y) = \lim\limits_{\D}\lim\limits_{\B} h(x,y)? 
    $$
\end{Problem}
Возможно, некоторые скажут, что эти пределы равны всегда, но это отнюдь не так. Хороший контрпример --- функция
\begin{gather*}
    h(x,y) =
    \begin{cases*}
        \cfrac{x^2-y^2}{x^2+y^2},\; \text{Если $(x,y) \neq 0$}\\
        0,\; \text{иначе}
    \end{cases*}
\end{gather*}
Для неё легко посчитать повторные пределы в нуле и показать, что они не равны. Действительно,
\begin{gather}
    \lim\limits_{y\to 0} h(x,y) =
    \begin{cases*}
        1,\; \text{если $x\neq 0$}\\
        -1,\; \text{иначе}
    \end{cases*}
\end{gather}
Тогда легко понять, что $\lim\limits_{x\to 0}\lim\limits_{y\to 0} h(x,y) = 1$. Аналогично показывается и что $\lim\limits_{y\to 0}\lim\limits_{x\to 0}h(x,y) = -1$.
\par Что же поможет нам идентифицировать такие ситуации? Поможет следующая теорема
\begin{Theorem}[Критерий Гордона]
    Следующие утверждения эквивалентны (внимание: здесь используются обозначения, аналогичные введённым ранее):
    \begin{enumerate}
        \item Повторные пределы $\lim\limits_{\B}f(x)$ и $\lim\limits_{\D} g(y)$ существуют и равны.
        \item $\forall \eps>0\; \exists B_\eps \in \B\colon\; \forall x\in B_\eps\; \exists D\in\D\colon \; \forall y \in D\; |h(x,y) -g(y)|<\eps$
    \end{enumerate}
\end{Theorem}
\begin{proof}[Доказательство]
    [(1) $\Rightarrow$ (2)] Пусть $\lim\limits_\B f(x) = \lim\limits_\D g(y) = A$. 
    \par Зафиксируем произвольное $\eps > 0$, $\eps_1 = \eps/3$.
    \begin{itemize}
        \item $\exists B_0\in \B\colon\; \forall x\in B_0\; |f(x) - A| <\eps_1$
        \item $\exists D_0\in \D\colon\; \forall y\in D_0\; |g(y) - A| <\eps_1$
    \end{itemize}
    В качестве $B_\eps$ возьмём $B_0$. Тогда
    $$
        \forall x\in B_0\; \exists \widetilde{D}_x \in \D\colon \forall y \in \widetilde{D}_x\; |h(x,y) -f(x)| < \eps_1
    $$
    $$
        \exists D_x \in \widetilde{D}_x\cap D_0
    $$
    Тогда $$
    \forall y\in D_x\; |h(x,y) - g(y)| \leqslant |h(x,y) - f(x)| + |f(x) - A| + |A- g(y)| < \eps_1 + \eps_1 + \eps_1 = \eps
    $$
    Получили требуемое.
    \par [$(2) \Rightarrow (1)$] Докажем для начала, что пределы есть. Зафиксируем произвольное $\eps > 0$, $\eps_1 = \eps/4$. Перепишем про ещё раз второй пункт
    $$
        \exists B_{\eps_1} \in \B\; \forall x\in B_{\eps_1}\; \exists D_x \in \D\colon\;  \forall y \in \D_x \; |h(x,y) - g(y)| < \eps_1
    $$
    Пусть $x_1, x_2 \in B_{\eps_1}$ --- произвольные.
    \begin{gather*}
        \exists D_{x_1} \in \D\colon \; \forall y\in D_{x_1}\colon \; |h(x_1,y) - g(y)|<\eps_1\\
        \exists D_{x_2}\in \D\colon \; \forall y \in D_{x_2}\colon \; |h(x_2, y) - g(y)| < \eps_1\\
        \exists \widetilde{D}_{x_1}\in \D\colon \; \forall y\in \widetilde{D}_{x_1}\colon \; |h(x_1,y) - f(x_1)| < \eps_1\\
        \exists \widetilde{D}_{x_2} \in \D\colon \; \forall y\in \widetilde{D}_{x_2}\colon \; |h(x_2,y) - f(x_2)| < \eps_1\\
    \end{gather*}
    Возьмём произвольное $y\in D_{x_1} \cap D_{x_2} \cap \widetilde{D}_{x_1}\cap \widetilde{D}_{x_2}$. Тогда
    \begin{gather}
        |f(x_1) - f(x_2)| \leqslant |f(x_1) - h(x_1, y)| + |h(x_1, y) - g(y)| + |g(y) - h(x_2, y)| + |h(x_2, y) - f(x_2)|<\\ < \eps_1 + \eps_1 + \eps_1 + \eps_1 = \eps
    \end{gather}
    Следовательно, по критерию Коши $\exists \lim\limits_{\B} f(x) = A$. Докажем, что $\exists \lim \limits_{\D} g(y) = A$. Зафиксируем произвольное $\eps > 0$.
    \par Найдём  $B_0 \in \B$ такое, что $\forall x\in B_0\; |f(x) - A| < \eps/3$. Найдём такое $B_{\eps/3} \in \B$ что 
    $$
        \forall x\in B_{\eps/3}\; \exists D_x\in \D\colon \; \forall y\in D_x\; |h(x,y) - g(y)| < \eps/3 
    $$
    Зафиксируем $x\in B_0\cap B_{\eps/3}$. Заметим, что 
    \begin{gather}
        \exists D_x \in \D\colon;\forall y\in D_x \; |h(x,y) - g(y)| <\eps/3\\
        \exists \widetilde{D}_x\in \D\colon \; \forall y\in \widetilde{D}_x\; |h(x,y) - f(x)| < \eps/3\\
        \ \\
        \exists D\in \D,\; D \subset D_x \cap \widetilde{D}_x\colon \; \forall y\in D\\
        |g(y) - A| \leqslant |g(y) - h(x,y)| + |h(x,y) - f(x)| + |f(x) - A| < \eps/3 + \eps/3 + \eps/3 = \eps
    \end{gather}
    Получили требуемое.
\end{proof}
\begin{Theorem}
    Пусть $X\subset \R$. $x_0$ --- его предельная точка (конечная или бесконечная). Пусть 
    $$
        \forall n \in \N\; \exists \lim\limits_{X\owns x \to x_0} f_n(x) = a_n
    $$
    а также $f_n(x)\overset{X}{\underset{n\to\infty}{\rightrightarrows}} f(x)$. Тогда  существуют и равны пределы $\lim\limits_{n \to \infty}a_n$ и $\lim\limits_{X \owns x\to x_0} f(x)$
\end{Theorem}
\begin{proof}
    Так как $f_n(x)\overset{X}{\underset{n\to\infty}{\rightrightarrows}} f(x)$, то
    $$
        \forall \eps >0\; \exists N\in \N\; \forall n>N\; \forall x \in X\; |f_n(x) - f(x)| < \eps
    $$
    Для существования предела необходимо и достаточно, чтобы
    $$
        \forall \eps > 0\; \exists N \in \N\; \forall n>N\; \exists \delta >0\; \forall x\in \delta(x_0)\cap X\; |f_n(x) - f(x)| < \eps
    $$
    Применяя критерий Гордона, получаем требуемое
\end{proof}
\begin{Consequence}
    Пусть $I$ --- невырожденный промежуток на $\R$ и для последовательности функций $f_n(x)$ известно, что $f_n(x) \in C(I)$ и $f_n(x)\overset{I}{\underset{n\to\infty}{\rightrightarrows}} f(x)$
    Тогда $f(x) \in C(I)$.
\end{Consequence}
\end{document}