\documentclass[a4paper, 12pt]{article}
\usepackage{header}

\begin{document}
\pagestyle{fancy}
\section{Лекция 03 от 26.09.2016}
\textbf{В прошлый раз} мы с вами сформулировали и доказали признак Лейбница. Будем пользоваться в этот раз слабой его формулировкой:
\begin{Statement}
	Пусть последовательность $\{b_n\}_{n = 1}^{\infty}$ монотонно убывает. Тогда ряд $\sum\limits_{n = 1}^{\infty}$ сходится. 
\end{Statement}
\begin{Comment}
	Для этого утверждения достаточно нестрогой монотонности.
\end{Comment}
Оказывается, этот признак является частным случаем более общего признака, который мы сейчас сформулируем и докажем.
\begin{Statement}[Признак Дирихле]
	Пусть частичные суммы ряда $\sum\limits_{n=1}^{\infty}a_n$ ограничены ($\exists C>0\colon \forall N\in \N \Rightarrow |A_n| = |a_1 + a_2 + \ldots + a_n| < C$), а $\{b_n\}_{n=1}^{\infty}$ --- неубывающая  последовательность. Тогда ряд $\sum\limits_{n=1}^{\infty}$ сходится. (отметим, что ряд $\sum\limits_{n=1}^{\infty}a_n$ может влёгкую расходиться). 
\end{Statement}
Подставляя сюда вместо $a_n$ последовательность $a_n = (-1)^n$ (частичные суммы $1, 0, 1, 0, \ldots$ ограничены), получаем формулировку первого утверждения.
\begin{Comment}
	В признаке Лейбница куда важнее утверждение про оценку остатка, чем про сходимость ряда.
\end{Comment}
\begin{proof}
	Для доказательства мы применим трюк, подобный интегрированию по частям, именуемый в дискретном случае \textit{преобразованием Абеля} (<<название умнее, чем само преобразование>> \textcopyright\ Лектор).
	\par  Зафиксируем произвольное $\varepsilon>0$. Найдём такое $N \in \N$, что 
	$$
		\forall n\geqslant N\colon\; b_n < \cfrac{\varepsilon}{4C}
	$$
	Возьмём $m>N$, произвольное $p\in \N$ и оценим сумму $\left|\sum\limits_{n=m + 1}^{m+p}a_n b_n\right|$.
	\begin{gather*}
		\sum\limits_{n=m+1}^{m+p} a_n b_n = \sum\limits_{n=m+1}^{m+p}\left(A_n - A_{n-1}\right)b_n = \sum\limits_{n=m+1}^{m+p} A_nb_n - \sum\limits_{n=m+1}^{m+p} A_{n-1} b_n = \sum\limits_{n=m+1}^{m+p}A_nb_n - \sum\limits_{n=m}^{m+p - 1}A_nb_{n-1} = \\
		= A_{m+p} b_{m+p} - A_mb_{m+1} + \sum\limits_{n=m+1}^{m+p - 1}A_n\left(b_n-b_{n+1}\right)
	\end{gather*}
	Следовательно, 
	\begin{gather*}
		\left|\sum\limits_{n=m+1}^{m+p} a_n b_n\right| \leqslant |A_{m+p}b_{m+p}| + |A_mb_{m+1}| + 		\sum\limits_{n=m+1}^{m+p-1} |A_n||b_n - b_{n+1}| <\\ < C\cfrac{\varepsilon}{4C} + C \cfrac{\varepsilon}{4C} + C 	\sum\limits_{n=m+1}^{m+p-1}( b_n - b_{n+1}) =\\= \cfrac{\varepsilon}{2} + C(b_{m+1} -b_{m+2} + b_{m + 2} - b_{m+3} + \ldots -b_{m+p}) = \cfrac{\varepsilon}{2} + C(b_{m+1} - b_{m+p}) < \cfrac{\varepsilon}{2} + C\cfrac{\varepsilon}{4c} < \varepsilon
	\end{gather*}
\end{proof}
	\begin{Examples}
		Попробуем поисследовать на сходимость какой-нибудь ряд, хороший пример --- ряд $\sum\limits_{n=1}^{\infty}\cfrac{\cos n\alpha}{n}$, при $\alpha \neq \pi k,\; k\in \Z$ или такой же с синусом. Пусть $a_n = \cos n\alpha $, $b_n = \cfrac{1}{n}$. Исследуем ряд $\sum\limits_{n=1}^{\infty} a_n$ на ограниченность частичных сумм:
		\begin{gather*}
			A_n = \cfrac{|\cos \alpha + \cos 2\alpha + \ldots + \cos n\alpha|\cdot \sin(\alpha/2)}{\sin(\alpha/2)} =\\=\left| \cfrac{-\sin\frac{\alpha}{2} + \sin\frac{3\alpha}{2}-\sin\frac{3\alpha}{2} + \sin\frac{5\alpha}{2} - \ldots + \sin\left(n + \frac{1}{2}\right)\alpha}{2\sin\frac{\alpha}{2}}\right| =\\
		= \cfrac{\left|-\sin \frac{\alpha}{2} + \sin\left(n + \frac{1}{2}\right)\alpha\right|}{2\left|\sin\frac{\alpha}{2}\right|} \leqslant \cfrac{2}{2\left|\sin\frac{\alpha}{2}\right|} = \cfrac{1}{\left|\sin\frac{\alpha}{2}\right|}
		\end{gather*}
		Тогда тут применим признак Дирихле и ряд сходится. 
		Теперь покажем его условную сходимость, то есть тот факт, что ряд из модулей расходится.
			\begin{gather}
			\cfrac{|\cos n\alpha|}{n} \geq \cfrac{(\cos n\alpha)^2}{n} = \cfrac{\cos 2n\alpha + 1}{2n}  = 
			\cfrac{1}{2}\left(\underbrace{\cfrac{\cos 2n\alpha}{n}}_{\text{сход.}} + \underbrace{\cfrac{1}{n}}_{\text{расх.}}\right)
				\end{gather}
	\end{Examples}
	Сформулируем и докажем ещё один важный признак.
	\begin{Statement}[Признак Абеля]
		Пусть $\sum\limits_{n=1}^{\infty}a_n$ сходится, а $b_n$ --- монотонная ограниченная последовательность. Тогда ряд $\sum\limits_{n=1}^{\infty}a_nb_n$ сходится.
	\end{Statement}
	\begin{proof}
		Последовательность частичных сумм ряда $\sum\limits_{n=1}^{\infty}$ сходится, а значит ограничена. Последовательность $b_n$ монотонна и ограничена, а значит имеет предел $B = \lim\limits_{n\to \infty} b_n$. То есть последовательность $b_n$ представима в виде $B + \beta_n$, где $\beta_n$ --- монотонно стремящаяся к нулю последовательность. 
		$$
		\sum\limits_{n=1}^{\infty}a_nb_n = \sum_{n=1}^{\infty} a_n(B+\beta_n) = \sum\limits_{n=1}^{\infty} a_nB + \sum_{n=1}^{\infty} a_n\beta_n
		$$
		Первое слагаемое сходится по условию (умножение на константу ничего не меняет), а второе по признаку Дирихле.
	\end{proof}
	Поговорим теперь о перестановках ряда.
	\begin{Def}
		Пусть $\sigma$ --- биекция (перестановка) $\N \to \N$. Говорят, что ряд $\sum\limits_{n=1}^{\infty}a_{\sigma(n)}$ является перестановкой ряда $\sum\limits_{n=1}^{\infty}a_n$.
	\end{Def}
	Сформулируем две теоремы, которые докажем в следующий раз.
	\begin{Theorem}[Коши]
		Пусть ряд $\sum\limits_{n=1}^{\infty}a_n$ абсолютно сходится, и его сумма равна $A$. Тогда $\forall \sigma$ --- перестановке $\N$ ряд $\sum\limits_{n=1}^{\infty}a_{\sigma(n)}$ также сходится абсолютно, и его сумма равна $A$.
	\end{Theorem}
	\begin{Theorem}[Римана]
		Пусть ряд $\sum\limits_{n=1}^{\infty}a_n$ сходится условно. Тогда\\
		\begin{enumerate}
			\item Для любого $A \in \mathbb{R}$ найдётся такая перестановка $\sigma$, что $\sum\limits_{n=1}^{\infty}a_{\sigma(n)} = A $
			\item Существует такая перестановка $\sigma$, что ряд $\sum\limits_{n=1}^{\infty}a_{\sigma(n)}$ расходится к $+\infty$.
			\item Существует такая перестановка $\sigma$, что ряд $\sum\limits_{n=1}^{\infty}a_{\sigma(n)}$ расходится к $-\infty$.
			\item Существует такая перестановка $\sigma$, что для ряда $\sum\limits_{n=1}^{\infty}a_{\sigma(n)}$ последовательность частичных сумм предела не имеет.
		\end{enumerate}
	\end{Theorem}
\end{document}
