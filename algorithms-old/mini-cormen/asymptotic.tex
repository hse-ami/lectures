\section{$O$-, $o$-, $\Omega$-, $\omega$-, $\Theta$- обозначения. Оценка сложности алгоритмов.}
\subsection{Асимптотические обозначения}

Введем следующие обозначения:

\[
  \Theta(g(n)) = \left\{ f(n)\mid \exists c_1>0, c_2>0 \exists n_0: \forall n 
  \geqslant n_0 \implies 0\leqslant c_1g(n)\leqslant f(n) \leqslant c_2g(n) \right\},
\]

$\Theta$ --- \emph{асимптотическое} $\mathcal{=}$. Например, $2n = \Theta(n)$. 
По определению, $c_1n \leqslant 2n \leqslant c_2n$. Тогда $c_1 = 1, c_2 = 2$.

\[
  O(g(n)) = \left\{ f(n)\mid \exists  c_2>0 \exists n_0: \forall n \geqslant n_0
  \implies 0\leqslant f(n) \leqslant c_2g(n) \right\}
\]

$O$ --- \emph{асимптотическое} $\leqslant$. Например, по этому определению 
$n = O(n \log{n})$, так как при достаточно больших $n_0$ следует, что
$\log n_0 > 1$. Тогда $c_2 = 1$.

\[
\Omega(g(n)) = \left\{ f(n)\mid \exists c_1>0 \exists n_0: \forall n \geqslant
 n_0 \implies 0\leqslant c_1g(n)\leqslant f(n) \right\}
\]

$\Omega$ --- \emph{асимптотическое} $\geqslant$. Например, $n \log n = \Omega(n 
\log n)$ и $n \log n = \Omega(n)$. В обоих случаях подходит $c_1 = 1$.

\[
  o(g(n)) = \left\{ f(n)\mid \forall  c_2>0 \exists n_0: \forall n \geqslant n_0
  \implies 0\leqslant f(n) \leqslant c_2g(n) \right\}
 \]

$o$ --- \emph{асимптотическое} $<$. Например, $n = o(n \log n)$. Покажем это. 
Пусть $n < c_2 n \log n \iff 1 < c_2 \log n \iff n > 2^{1/c_2}$. Тогда $n_0 = [2^{1/c_2} + 1]$

\[
  \omega(g(n)) = \left\{ f(n)\mid \forall c_1>0 \exists n_0: \forall n \geqslant
  n_0 \implies 0\leqslant c_1g(n)\leqslant f(n) \right\}
\]

$\omega$ --- \emph{асимптотическое} $>$. Например, нельзя сказать, что $n \log n
 = \omega(n \log n)$. Но можно сказать, что  $n \log n = \omega(n)$.

Заметим, что в логарифмах можно свободно менять основание: $\log_c n = \frac 
{\log_2 n}{\log_2 c}$. Именно поэтому не пишут основание логарифма.
