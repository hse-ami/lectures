% Размер страницы и шрифта
\documentclass[12pt,a4paper]{article}

%% Работа с русским языком
\usepackage{cmap}                   % поиск в PDF
\usepackage{mathtext}               % русские буквы в формулах
\usepackage[T2A]{fontenc}           % кодировка
\usepackage[utf8]{inputenc}         % кодировка исходного текста
\usepackage[english,russian]{babel} % локализация и переносы

%% Изменяем размер полей
\usepackage[top=0.5in, bottom=0.75in, left=0.625in, right=0.625in]{geometry}

%% Различные пакеты для работы с математикой
\usepackage{mathtools}  % Тот же amsmath, только с некоторыми поправками
\usepackage{amssymb}    % Математические символы
\usepackage{amsthm}     % Пакет для написания теорем
\usepackage{amstext}
\usepackage{array}
\usepackage{amsfonts}
\usepackage{icomma}     % "Умная" запятая: $0,2$ --- число, $0, 2$ --- перечисление

%% Графика
\usepackage[pdftex]{graphicx}
\graphicspath{{images/}}

%% Прочие пакеты
\usepackage{listings}               % Пакет для написания кода на каком-то языке программирования
\usepackage{algorithm}              % Пакет для написания алгоритмов
\usepackage[noend]{algpseudocode}   % Подключает псевдокод, отключает end if и иже с ними
\usepackage{indentfirst}            % Начало текста с красной строки
\usepackage[colorlinks=true, urlcolor=blue]{hyperref}   % Ссылки
\usepackage{pgfplots}               % Графики
\pgfplotsset{compat=1.12}
\usepackage{forest}                 % Деревья
\usepackage{titlesec}               % Изменение формата заголовков
\usepackage[normalem]{ulem}         % Для зачёркиваний
\usepackage[autocite=footnote]{biblatex}    % Кавычки для цитат и прочее
\usepackage[makeroom]{cancel}       % И снова зачёркивание (на этот раз косое)

% Изменим формат \section и \subsection:
\titleformat{\section}
	{\vspace{1cm}\centering\LARGE\bfseries} % Стиль заголовка
	{}                                      % префикс
	{0pt}                                   % Расстояние между префиксом и заголовком
	{}                                      % Как отображается префикс
\titleformat{\subsection}                   	% Аналогично для \subsection
	{\Large\bfseries}
	{}
	{0pt}
	{}

% Поправленный вид lstlisting
\lstset { %
    backgroundcolor=\color{black!5}, % set backgroundcolor
    basicstyle=\footnotesize,% basic font setting
}

% Теоремы и утверждения. В комменте указываем номер лекции, в которой это используется.
\newtheorem*{hanoi_recurrent}{Свойство} % Лекция 1
\let\epsilent\varepsilon                % Лекция 8
\DeclareMathOperator{\rk}{rank}         % Лекция 20

% Информация об авторах
\author{Группа лектория ФКН ПМИ 2015-2016 \\
	Никита Попов \\
	Тамерлан Таболов \\
	Лёша Хачиянц}
\title{Лекции по предмету \\
	\textbf{Алгоритмы и структуры данных}}
\date{2016 год}


\begin{document}

\section*{Лекция 19 от 22.03.2016}

\subsection{Хорошая хеш-функция}

Пусть $H$ --- семейство хеш-функций $h: U \to \{0, 1, \ldots, m-1\}$. Назовём $H$ \emph{универсальной}, если выплолняется
\[\forall x\in U, y\in U \left|\left\{ h(x) = h(y)\mid h\in H \right\}\right| = \frac{|H|}{m}\]

При случайном выборе $h\in H$ $\mathrm{Pr}[h(x) = h(y)] = \frac{1}{m}$

Будем считать, что пользуемся такой $h$.

Пусть $m = n^2$, где $n$ --- число хранимых ключей. Выберем $h\in H$; сколько будет коллизий (таких пар $(x, y)$, что $h(x) = h(y)$)? Утверждается, что немного.

Давайте оценим матожидание коллизии:

\[
    E(\text{число коллизий}) = Pr(\text{коллизия})*{n \choose 2} = \frac{1}{m}\cdot\frac{n(n-1)}{2} = \frac{n-1}{2n}<\frac{1}{2}
\]

\[
    Pr(\text{\# коллизий} \geqslant 1) \leqslant \frac{ E(\text{\# коллизий})}{1} = \frac{1}{2}
\]

\[
    Pr(\text{коллизий нет}) \geqslant \frac{1}{2}
\]

Заметим, что тогда (мы опираемся на то, что данные сохраняются единожды) получается, что потратив в среднем две попытки мы можем найти такую $h$, что хеширование произойдёт без коллизий и поиск будет за константное время (это, кстати, называется \emph{идеальным хешированием}).

Или давайте так:

Создадим таблицу из $m = n$ ячеек; но таблица будет не простой, а состоящей из хеш-таблиц; при этом внутри каждой такой таблицы коллизий не будет (мы об этом позаботимся).

Пусть $n_j$ --- число элементов таких, что $h(x) = j$. $\sum\limits_j n_j = n$, как несложно заметить. Введём также $m_j$ --- число ячеек в $j$-ой таблице второго уровня. Если мы хотим обеспечить отсутствие коллизий (и не потратить на это кучу времени), то $m_j$ должно быть равно $n_j^2$. Вопрос --- а чем такой способ лучше предыдущего? А давайте посмотрим на память: внешняя таблица имеет линейное количество ячеек, где каждая имеет константную память (там хранятся лишь ссылки на таблицы второго уровня). А память на таблицы второго уровня --- $\sum\limits_{j=0}^{m-1} \Theta(n_j^2)$. Понятно, что если придумать худший случай, то все будут в одной ячейке. Давайте тогда считать матожидание:

\[
    E\left[ \sum\limits_{j=0}^{m-1} \Theta(n_j^2) \right]
\]

Но сначала вот что:

\[
    n_j^2 = n_j +n_j^2 - n_j = n_j+n_j(n_j-1) = n_j + 2{n_j \choose 2}
\]

\begin{multline*}
    E\left[ \sum\limits_{j=0}^{m-1} n_j^2 \right] = 
    E\left[ \sum\limits_{j=0}^{m-1} n_j + 2{n_j \choose 2} \right] = 
    E\left[ \sum\limits_{j=0}^{m-1} n_j \right] + 2E\left[{n_j \choose 2} \right] = \\ =
    n + 2E\left[ \text{\# коллизий для $h$} \right] \leqslant
    n + 2\cdot\frac{1}{m}\cdot\frac{n(n-1)}{2} =
    n + n - 1 =
    2n - 1
\end{multline*}

Ой! Линейная память! Вот и искомое преимущество.

Но пока всё довольно абстрактно. Где брать такие универсальные хеш-функции? Давайте приведём пример:

Рассмотрим $H_{p\, m}$, где $p$ --- простое число, большее всех ключей.
Пусть она состоит из функций $h(k) = \left( \left( ak+b \right)\mod{p} \right)\mod{m};\ a\in[1;p],\ b\in[0; p]$

\subsection{Красно-чёрное дерево}

Что такое бинарное дерево поиска мы знаем; рассмотрим теперь \emph{сбалансированное} бинарное дерево. Его ключевое свойство: высота такого дерева --- $O(\log n)$. Разумеется, речь идёт о семействе деревьев; сказать, выполняется ли это для конкретного дерева нельзя.

Рассмотрим один класс таких деревьев --- \emph{красно-чёрные} деревья. Ключевые характеристики:

\begin{enumerate}
    \item Любой узел --- красный или чёрный.
    \item Корень и листья --- чёрные.
    \item Родителем красного узла может быть только чёрный.
    \item На всех простых путях из узла $x$ до листьев-наследников одинаковое количество чёрных узлов.
\end{enumerate}

Вот как может выглядеть такое дерево:

\includegraphics[width=15cm]{images/19_rb_tree.pdf}

Высота RB-дерева с $n$ ключами не больше $2\log_2(n+1)$. Докажем это не вполне формально:

``Подтянем'' красные узлы к чёрным родителям. Легко заметить, что высота дерева уменьшилась не больше, чем вдвое: $n \leqslant 2h'$. Число листьев, получившихся в итоге --- $n+1 \geqslant 2^{h'}$; $\log_2(n+1)\geqslant h'\geqslant\frac{h}{2}$. Значит, высота дерева --- $O(\log n)$ и поиск в дереве займёт логарифмическое время.

\begin{lstlisting}
rb_insert(t, x):
    tree_insert(t, x)
    x.color = 'red'
    while x != t.root and x.p.color = 'red' do
        if x.p = x.p.p.right then
            y:= x.p.p.left
            if y.color = 'red' then
                x.p.color:= 'black'
                y.color:= 'black'
                x.p.p.color = 'red'
            else
                AAAAAAAAAAAA
\end{lstlisting}
\end{document}
