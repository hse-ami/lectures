% Размер страницы и шрифта
\documentclass[12pt,a4paper]{article}

%% Работа с русским языком
\usepackage{cmap}                   % поиск в PDF
\usepackage{mathtext}               % русские буквы в формулах
\usepackage[T2A]{fontenc}           % кодировка
\usepackage[utf8]{inputenc}         % кодировка исходного текста
\usepackage[english,russian]{babel} % локализация и переносы

%% Изменяем размер полей
\usepackage[top=0.5in, bottom=0.75in, left=0.625in, right=0.625in]{geometry}

%% Различные пакеты для работы с математикой
\usepackage{mathtools}  % Тот же amsmath, только с некоторыми поправками
\usepackage{amssymb}    % Математические символы
\usepackage{amsthm}     % Пакет для написания теорем
\usepackage{amstext}
\usepackage{array}
\usepackage{amsfonts}
\usepackage{icomma}     % "Умная" запятая: $0,2$ --- число, $0, 2$ --- перечисление

%% Графика
\usepackage[pdftex]{graphicx}
\graphicspath{{images/}}

%% Прочие пакеты
\usepackage{listings}               % Пакет для написания кода на каком-то языке программирования
\usepackage{algorithm}              % Пакет для написания алгоритмов
\usepackage[noend]{algpseudocode}   % Подключает псевдокод, отключает end if и иже с ними
\usepackage{indentfirst}            % Начало текста с красной строки
\usepackage[colorlinks=true, urlcolor=blue]{hyperref}   % Ссылки
\usepackage{pgfplots}               % Графики
\pgfplotsset{compat=1.12}
\usepackage{forest}                 % Деревья
\usepackage{titlesec}               % Изменение формата заголовков
\usepackage[normalem]{ulem}         % Для зачёркиваний
\usepackage[autocite=footnote]{biblatex}    % Кавычки для цитат и прочее
\usepackage[makeroom]{cancel}       % И снова зачёркивание (на этот раз косое)

% Изменим формат \section и \subsection:
\titleformat{\section}
	{\vspace{1cm}\centering\LARGE\bfseries} % Стиль заголовка
	{}                                      % префикс
	{0pt}                                   % Расстояние между префиксом и заголовком
	{}                                      % Как отображается префикс
\titleformat{\subsection}                   	% Аналогично для \subsection
	{\Large\bfseries}
	{}
	{0pt}
	{}

% Поправленный вид lstlisting
\lstset { %
    backgroundcolor=\color{black!5}, % set backgroundcolor
    basicstyle=\footnotesize,% basic font setting
}

% Теоремы и утверждения. В комменте указываем номер лекции, в которой это используется.
\newtheorem*{hanoi_recurrent}{Свойство} % Лекция 1
\let\epsilent\varepsilon                % Лекция 8
\DeclareMathOperator{\rk}{rank}         % Лекция 20

% Информация об авторах
\author{Группа лектория ФКН ПМИ 2015-2016 \\
	Никита Попов \\
	Тамерлан Таболов \\
	Лёша Хачиянц}
\title{Лекции по предмету \\
	\textbf{Алгоритмы и структуры данных}}
\date{2016 год}


\begin{document}

\section*{Лекция 20 от 24.03.2016}

\subsection{Вставка красно-чёрные деревья}

В прошлый раз мы получили, что высота красно-чёрного дерева --- $O(\log n)$, где $n$ --- число ключей.

Когда мы вставили новую вершину, мы назначаем ей красный цвет. При этом может нарушиться одно из свойств --- у красного только чёрный родитель.

Править мы будем, используя только две операции:
\begin{enumerate}
    \item Перекрашивание --- смена цвета на противоположный.
    \item Поворот --- за картинкой полезайте в Кормена, в общем.
\end{enumerate}

Вообще, тут куча графов, я не успеваю перерисовывать, я сфоткал, если что.

Сложность такого алгоритма --- $O(\log n)$ --- на изначальную вставку уходит $O(\log n)$; ошибки двигаются вверх не больше чем за два поворота и ?? перекрашиваний --- тоже $O(\log n)$.

\subsection{Самоорганизующийся список}
Представьте, что вы готовитесь к экзамену. У вас есть стопка книг (все в белых обложках) и вы ищете нужную вам, а найдя --- кладёте вверх.

Формализуем: есть список $L$. Есть операции access$(x, L)$ за rank$(x)$; transpose$(x, y)$ --- за $O(1)$, но меняем только соседние элементы.

Решать будем для онлайн-алгоритма --- он не знает всех наших будущих запросов.

Пусть $|L| = n$, $S$ --- множество операций.

Понятно, что сложность в худшем случае не может быть меньше $\Theta(n|S|)$, то есть $c_A(S) = \Omega(n|S|)$

Средний случай: пусть $p(x)$ --- вероятность обращения к $x$.

$E(c_A(S)) = \sum\limits_{x\in L}\left(p(x)\rk(x)\right)$

Эвристика MTF --- после доступа помещаем элемент вверх списка. Оценим MTF:

Для начала введём понятие $\alpha$-конкуррентности; алгоритм $\alpha$-конкуррентен, если $C_A(S) \leqslant \alpha\cdot C_{OPT}(S)+k$.

Пусть $L_i^A$ --- список после $i$ операций по алгоритму $A$; $C_i^A$ --- стоимость $i$-ой операции по алгоритму $A$.

$\Phi(L_i^{MTF}) = 2|\{(x, y)\mid\rk_{L_i^{MTF}}(x) < \rk_{L_i^{MTF}}(y),\ \rk_{L_i^{OPT}}(x) > \rk_{L_i^{OPT}}(y) \}|$

При этом после одной транспозиции $\Phi$ меняется на $\pm 2$
\end{document}
