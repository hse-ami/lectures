\documentclass[a4paper, 12pt]{article}
%%% Работа с русским языком
\usepackage{cmap}                   % поиск в PDF
\usepackage{mathtext}               % русские буквы в формулах
\usepackage[T2A]{fontenc}           % кодировка
\usepackage[utf8]{inputenc}         % кодировка исходного текста. НИКОГДА НЕ МЕНЯТЬ.
\usepackage[english,russian]{babel} % локализация и переносы
\usepackage{epigraph}               % делать эпичные эпиграфы
\usepackage{fancybox,fancyhdr}      % для колонтитулов

%% Отступы между абзацами и в начале абзаца 
\setlength{\parindent}{0pt}
\setlength{\parskip}{\medskipamount}

%% Изменяем размер полей
\usepackage[top=0.7in, bottom=0.75in, left=0.625in, right=0.625in]{geometry}

%% Изменяем размер отступов колонтитулов
\renewcommand{\headrulewidth}{1.8pt}
\renewcommand{\footrulewidth}{0.0pt}

%% Графика
\usepackage[pdftex]{graphicx}
\graphicspath{{images/}}

%% Различные пакеты для работы с математикой
\usepackage{mathtools}      % Тот же amsmath, только с некоторыми поправками

\usepackage{amssymb}        % Математические символы
\usepackage{amsthm}         % Пакет для написания теорем
\usepackage{amstext}
\usepackage{array}
\usepackage{amsfonts}
\usepackage{icomma}         % "Умная" запятая: $0,2$ --- число, $0, 2$ --- перечисление
\usepackage{enumitem}       % Для выравнивания itemize (\begin{itemize}[align=left])

% Номера формул
\mathtoolsset{showonlyrefs=true} % Показывать номера только у тех формул, на которые есть \eqref{} в тексте.

% Ссылки
\usepackage[colorlinks=true, urlcolor=blue]{hyperref}

% Шрифты
\usepackage{euscript}  % Шрифт Евклид
\usepackage{mathrsfs}  % Красивый матшрифт

% Свои команды\textbf{}
\DeclareMathOperator{\sgn}{\mathop{sgn}}

% Перенос знаков в формулах (по Львовскому)
\newcommand*{\hm}[1]{#1\nobreak\discretionary{}
{\hbox{$\mathsurround=0pt #1$}}{}}

% Графики
\usepackage{tikz}
\usepackage{pgfplots}
%\pgfplotsset{compat=1.12}

% Изменим формат \section и \subsection:
\usepackage{titlesec}
\titleformat{\section}
{\vspace{1cm}\centering\LARGE\bfseries} % Стиль заголовка
{}                        % префикс
{0pt}                     % Расстояние между префиксом и заголовком
{}                        % Как отображается префикс
\titleformat{\subsection} % Аналогично для \subsection
{\Large\bfseries}
{}
{0pt}
{}


%% Информация об авторах
\author{Группа лектория ФКН ПМИ 2016-2017 \\
  Алексей Хачиянц \\
  Никита Попов \\
  Денис Беляков}
\title{Лекции по предмету \\
  \textbf{Теория вероятностей и математическая статистика}}
\date{2016/2017 учебный год}


%% Для колонтитула
\def\head{
{\it \small НИУ ВШЭ $\bullet$ Факультет компьютерных наук $\bullet$
Прикладная математика и информатика}
}

%% Делаем верхний и нижний колонтитулы
\fancyhf{}
\fancyhead[R]{\head}
\fancyfoot[R]{{\small {\it Алексей Хачиянц,
  Никита Попов,
  Денис Беляков. Теория вероятностей и математическая статистика}}}
\fancyhead[L]{\thepage}

%% Функция для обнуление конкретного счётчика
\def\ob#1{\setcounter{#1}{0}}

%% Функция для обнуления всего. Обязательно вызывать в конце файла!
\def\oball{\setcounter{Def}{0}\setcounter{Lem}{0}\setcounter{Sug}{0}
\setcounter{eg}{0}\setcounter{Cmt}{0}\setcounter{cnsqnc}{0}\setcounter{th}{0}
\setcounter{stnmt}{0}\setcounter{task}{0}\setcounter{dsg}{0}\setcounter{gen}{0} 
\setcounter{dream}{0}\setcounter{prop}{0}}

%% Определение счётчика + переопределение команд

\newcounter{Def}\setcounter{Def}{0}
\def\df{\par\smallskip\refstepcounter{Def}\textbf{\arabic{Def}}}
\newtheorem*{Def}{Определение \df}

\newcounter{Lem}\setcounter{Lem}{0}
\def\lm{\par\smallskip\refstepcounter{Lem}\textbf{\arabic{Lem}}}
\newtheorem*{Lemma}{Лемма \lm}

\newcounter{Sug}\setcounter{Sug}{0}
\def\sug{\par\smallskip\refstepcounter{Sug}\textbf{\arabic{Sug}}}
\newtheorem*{Suggestion}{Предложение \sug}

\newcounter{eg}\setcounter{eg}{0}
\def\eg{\par\smallskip\refstepcounter{eg}\textbf{\arabic{eg}}}
\newtheorem*{Examples}{Пример \eg}

\newcounter{Cmt}\setcounter{Cmt}{0}
\def\cmt{\par\smallskip\refstepcounter{Cmt}\textbf{\arabic{Cmt}}}
\newtheorem*{Comment}{Замечание \cmt}

\newcounter{cnsqnc}\setcounter{cnsqnc}{0}
\def\cnsqnc{\par\smallskip\refstepcounter{cnsqnc}\textbf{\arabic{cnsqnc}}}
\newtheorem*{Consequence}{Следствие \cnsqnc}

\newcounter{th}\setcounter{th}{0}
\def\th{\par\smallskip\refstepcounter{th}\textbf{\arabic{th}}}
\newtheorem*{Theorem}{Теорема \th}

\newcounter{stnmt}\setcounter{stnmt}{0}
\def\st{\par\smallskip\refstepcounter{stnmt}\textbf{\arabic{stnmt}}}
\newtheorem*{Statement}{Утверждение \st}

\newcounter{task}\setcounter{task}{0}
\def\task{\par\smallskip\refstepcounter{task}\textbf{\arabic{task}}}
\newtheorem*{Task}{Упражнение \task}

\newcounter{dsg}\setcounter{dsg}{0}
\def\dsg{\par\smallskip\refstepcounter{dsg}\textbf{\arabic{dsg}}}
\newtheorem*{Designation}{Обозначение \dsg}

\newcounter{gen}\setcounter{gen}{0}
\def\gen{\par\smallskip\refstepcounter{gen}\textbf{\arabic{gen}}}
\newtheorem*{Generalization}{Обобщение \gen}

\newcounter{dream}\setcounter{dream}{0}
\def\dream{\par\smallskip\refstepcounter{dream}\textbf{\arabic{dream}}}
\newtheorem*{Thedream}{Предел мечтаний \dream}

\newcounter{prop}\setcounter{prop}{0}
\def\prop{\par\smallskip\refstepcounter{prop}\textbf{\arabic{prop}}}
\newtheorem*{Properties}{Свойства \prop}

\renewcommand{\Re}{\mathrm{Re\:}}
\renewcommand{\Im}{\mathrm{Im\:}}
\newcommand{\Arg}{\mathrm{Arg\:}}
\renewcommand{\arg}{\mathrm{arg\:}}
\newcommand{\Mat}{\mathrm{Mat}}
\newcommand{\id}{\mathrm{id}}
\newcommand{\isom}{\xrightarrow{\sim}} 
\newcommand{\leftisom}{\xleftarrow{\sim}}
\newcommand{\Hom}{\mathrm{Hom}}
\newcommand{\Ker}{\mathrm{Ker}\:}
\newcommand{\rk}{\mathrm{rk}\:}
\newcommand{\diag}{\mathrm{diag}}
\newcommand{\ort}{\mathrm{ort}}
\newcommand{\pr}{\mathrm{pr}}
\newcommand{\vol}{\mathrm{vol\:}}
\def\limref#1#2{{#1}\negmedspace\mid_{#2}}
\newcommand{\eps}{\varepsilon}

\renewcommand{\phi}{\varphi} % плохо, так как есть \phi в англ раскладке.
\newcommand{\e}{\mathbb{e}}
\renewcommand{\l}{\lambda}
\newcommand{\N}{\mathbb{N}}
\newcommand{\Z}{\mathbb{Z}}
\newcommand{\Q}{\mathbb{Q}}
\newcommand{\R}{\mathbb{R}}
\renewcommand{\C}{\mathbb{C}}
\newcommand{\E}{\mathbb{E}}

\newcommand{\vvector}[1]{\begin{pmatrix}{#1}_1 \\\vdots\\{#1}_n\end{pmatrix}}
\renewcommand{\vector}[1]{({#1}_1, \ldots, {#1}_n)}
%\usepackage{algorithm}
%\usepackage[noend]{algpseudocode}


\begin{document}
%\maketitle

\section{Лекция 3. Безусловная многомерная минимизация}

Рассмотрим следующую задачу минимизации функции: $$f(x) \rightarrow \min_{x \in \R^n}, f \in C^1, O(x) = \{f(x), \nabla f(x)\}$$

Напомним, что $O(x)$ - оракул, то есть совокупность величин, которые возможно вычислить в заданной точке $x$ (см. лекцию 2).

Из методов многомерной оптимизации можно выделить два вида: 

\begin{enumerate}
    \item Методы поиска по направлению (Line Search). 
    \item Методы доверительной области (Trust Region).
\end{enumerate}

\subsection{Методы поиска по направлению (Line Search)}

В методах поиска по направлению на каждом шаге значение рассматриваемой точки - кандидата на локальный минимум - изменяется по следующему правилу:
$$x_{k+1} = x_k + \alpha_{k}d_{k}$$
$d_{k} \in \R^n$ - \textit{вектор спуска}, то есть для функции $\phi(\alpha) &= f(x_k + \alpha d_k)$ выполняется условие: 
$$\phi'(0) &=\nabla f(x_k)^T d_k  < 0$$

$\alpha_{k} \in \R_+ $ - длина шага. Для нахождения $\alpha_{k}$ решается задача одномерной оптимизации:
$$\alpha_k = \argmin_\alpha f(x_k + \alpha d_k)$$

\subsection{Методы доверительной области (Trust Region)}

В методах доверительной области следующая точка изменяется по правилу $x_{k+1} = x_k + d_k$. Для нахождения $d_k$ решается система уравнений:
\begin{align*}
\begin{cases}
        f(x_k + d_k) \approx m_k(d_k) = f(x_k) + g_k^Td_k+\frac{1}{2}d_k^TB_kd_k \rightarrow \min_{d_k}\\
        ||d_k|| \leq \Delta_k
\end{cases}
\end{align*}

Здесь $\Delta_k$ - радиус доверительной области, $g_k = \nabla f(x_k)$, $B_k$ - гессиан или его аппроксимация.

Второе уравнение задает доверительную область (trust region). Заметим, что минимизация уже не глобальная, а на доверительной области.

Trust Region - более гибкий подход. При изменении радиуса доверительной области направление $d_k$ может измениться. При увеличении радиуса качество подбора направления возрастает. Однако при использовании такого подхода на каждом шаге нужно решать многомерную задачу оптимизации, в то время как Line search сводится к одномерной. 

%TODO: картика, иллюстрирующая изменение направления d_k от величины окрестности

На практике Trust Region используют при "дорогом" оракуле. В таком случае оракул запускается меньшее количество раз, на каждом шаге задача оптимизации считается более качественно. 

\subsection{Вычисление длины шага в линейном поиске}

Вернемся к линейному поиску и рассмотрим несколько методов вычисления длины шага $\alpha_k$. Введем функцию $\phi(x) = f(x_k + \alpha_k d_k)$, чтобы $d_k$ был вектором спуска должно выполняться условие $\phi'(0) < 0$.

\subsubsection*{Условие Армихо}

Условием Армихо называется выполнение неравенства:
$$\phi(\alpha_k) \leq \phi(0) + c_1\alpha_k\phi'(0), \quad c_1 \in (0,1) $$

\subsubsection*{Условия Вульфа}

Условия Вульфа являются дополнениями к условию Армихо и дополнительно требуют следующее:

Слабое условие Вульфа:
$$\phi'(\alpha_k) \geq c_2 \phi'(0) $$
Сильное условие Вульфа:
$$|\phi'(\alpha_k)| \leq c_2 |\phi'(0)| $$

Здесь $c_2 \in (c_1,1)$.

\begin{Statement}
Пусть $\phi \in C^1$ - непрерывно-дифференцируема и для любого $\alpha$ выполняется $\phi(\alpha) > -\infty$. Тогда для любых $c_1$ и $c_2$: $0 < c_1 \leq c_2 < 1$ найдется $\alpha_k$, удовлетворяющая условию Вульфа.
\end{Statement}

На практике берут $c_1 = 10^{-4}$ и $c_2 = 0.9$ и $0.1$ - для достаточно грубой и более точной одномерной оптимизации соответственно.

%TODO: картиночка с иллюстрацией условия Армихо, слабого и сильного Вульфа

Чтобы найти $\alpha_k$, удовлетворяющую условиям, используем метод, аналогичный методу Брента. Алгоритм можно разбить на 2 шага:

\begin{enumerate}
    \item Этап расширения: расширение интервала, до тех пор, пока не будут выполнены условия нахождения искомой точки в рассматриваемом интервале.
    \item Этап сужения: ищем точку внутри (например, с помощью метода золотого сечения) и уменьшаем интервал. 
\end{enumerate}

Для нахождения точки, удовлетворяющей условию Армихо, используют метод дробления шага:

\begin{algorithm}
\caption{Процедура backtracking}\label{alg:backtrack}
\begin{algorithmic}[1]
\State $\alpha \gets \alpha_0$
\Loop
    \If {$\phi(\alpha) \leq \phi(0) + \alpha c_1 \phi'(0)$} \Then
    \State\textbf{break}
    \EndIf
    \State $\alpha \gets \alpha \rho$, $\rho < 1$
\EndLoop
\end{algorithmic}
\end{algorithm}

Теперь построим метод, выбирающий вектор спуска $d_k$.

\subsection{Метод градиентного спуска}

Сформулируем задачу линейного поиска так:
$$x_{k+1} = x_k + \alpha_k d_k$$
\begin{align*}
\begin{cases}
        \nabla f(x_k)^Td_k \rightarrow \min_{d_k}\\
        ||d_k||^2 \leq 1
\end{cases}
\end{align*}

Выразим отсюда направление спуска:
$$d_k = -\frac{\nabla f(x_k)}{||\nabla f(x_k)||}$$

Полученное $d_k$ соответствует выбору направления на шаге градиентного спуска:
$$x_{k+1} = x_k - \alpha_k \nabla f(x_k)$$

Стратегии выбора длины шага $\alpha_k$:

\begin{enumerate}
    \item $\alpha_k = \frac{1}{L}$, где L - константа Липшица, $f \in C^{1,1}_L$
    \item  $\alpha_k$ удовлетворяет условиям Армихо-Вольфа для заданных $c_1, c_2$
\end{enumerate}

Рассмотрим более подробно первую стратегию. Если функция $f \in C^{1,1}_L$, то ее можно ограничить параболой.
$$f(y) \leq f(x) + \nabla f(x)^T (y - x) +\frac{L}{2} ||y - x||^2 = (*) $$

Попробуем оценить функцию глобально. Вновь воспрользуемся тем, что функция Липшицева, а значит, ее градиент ограничен. Пусть $y = x - \alpha \nabla f(x)$, тогда:
$$(*) = f(x) -\alpha ||\nabla f(x)||^2 + \frac{L \alpha ^2}{2} ||\nabla f(x)||^2 = f(x) - \alpha(1-\frac{L \alpha}{2})||\nabla f(x)||^2$$

Возьмем $\alpha$ такое, чтобы коэффициент при норме градиента был наибольшим: 
$$\alpha\left(1-\frac{L \alpha}{2}\right) \rightarrow \max_{\alpha}$$
Тогда $\alpha_{opt} = \frac{1}{L}$ и $\alpha_{opt}\left(1-\frac{L \alpha_{opt}}{2}\right)  = \frac{1}{2L}$

Значит, гарантированный прогресс на итерации работы алгоритма равен $\frac{1}{2L}||\nabla f(x)||^2$. 

%TODO: картиночка

$$f(x_{k+1}) \leq f(x_{k}) - \frac{1}{2L}||\nabla f(x_k)||^2 \leq f(x_{k-1}) - \frac{1}{2L}||\nabla f(x_{k-1})||^2 - \frac{1}{2L}||\nabla f(x_{k})||^2 \leq \dots$$
$$\dots \leq f(x_{0}) - \frac{1}{2L}\sum_{i = 0}^{k}||\nabla f(x_i)||^2$$
$$\sum_{i = 0}^{k}||\nabla f(x_i)||^2 \leq 2L(f(x_0) - f(x_{k+1})) \leq 2L(f(x_0) - f_{opt})$$

При $ k \rightarrow \infty$ ряд сходится, значит, $||\nabla f(x_{k-1})||^2 \rightarrow 0$

Подсчитаем скорость, с которой $\nabla f(x_{k-1})$ стремится к 0:

$$g_k = \min_{0\leq i\leq k}||\nabla f(x_{i})||^2$$

$$(k+1)g_k \leq \sum_{i = 0}^{k}||\nabla f(x_i)||^2 \leq 2L(f(x_0) - f_{opt})$$

\begin{Statement}
Если $ f \in C^{1,1}_L$ и $\forall x f(x) > -\infty $, то для градиентного спуска с $\alpha_k = \frac{1}{L}$ верно, что $g_k \leq \frac{c}{k+1}$, где c - некоторая константа.
\end{Statement}

Попробуем усилить предположение. Добавим требование к выпуклости функции. Пусть $ f \in C^{1,1}_L$ и $f$-сильно выпуклая с параметром $\mu$. По определению сильной выпуклости:
$$ f(y) \geq f(x) + \nabla f(x)^T (y - x) + \frac{\mu}{2}\| y - x \|^2 = g(y)$$
$$f_{opt} = \min_y f(y) \geq \min_y g(y) $$
$$\nabla_y g(y) = \nabla f(x) + \mu(y - x) = 0$$
$$y_{opt} = x - \frac{1}{\mu}\nablaf(x), \quad g(y_{opt}) = f(x) - \frac{1}{2\mu} \| \nabla f(x) \|^2$$

$$f_{opt} \geq f(x) - \frac{1}{2\mu} \| \nabla f(x) \|^2 \quad \rightarrow \quad \| \nabla f(x) \|^2 \geq 2\mu(f(x) - f_{opt}) $$

Объединим полученные неравенства:
$$f(x_k) - f_{opt} \leq f(x_k) - f_{opt} - \frac{1}{2L} \| \nabla f(x) \|^2 \leq f(x_k) - f_{opt} - \frac{\mu}{L} (f(x) - f_{opt}) = (1 - \frac{\mu}{L})(f(x_0) - f_{opt})$$

\begin{Statement}
Если $ f \in C^{1,1}_L$, и $f$ - сильно выпуслая с параметром $\mu$, то для градиентного спуска с $\alpha_k = \frac{1}{L}$ верно, что $f(x_k) - f{opt} \leq (1 - \frac{\mu}{L}(f(x_0 - f_{opt})$.
\end{Statement}

\begin{table}[h]
    \centering
    \begin{tabular}{|c|c|c|} 
        \hline
        Функция & Невязка $r_k$ & Скорость сходимости\\
        \hline
        \hline
        $f \in C^{1,1}_L $ & $\min_{0\leq i \leq k}\|\nabla f(x_i)\|^2$& $O(\frac{1}{k})$\\
        \hline
        $f \in C^{1,1}_L$, $f$ - выпуклая & $f(x_k) - f_{opt}$& $O(\frac{1}{k})$\\
        \hline
        $f \in C^{1,1}_L$, $f$ - $\mu$-сильно выпуклая & $f(x_k) - f_{opt}$& $O(c^k),\: c = 1 - \frac{\mu}{L}$\\
        \hline
    \end{tabular}
    \caption{Скорость сходимости градиентного спуска}
\end{table}


\end{document}