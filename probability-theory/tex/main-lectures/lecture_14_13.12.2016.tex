\section{Лекция от 13.12.2016}
\subsection{Три неравенства}
Для случайных величин в общем случае выполнены те же неравенства, что и для дискретного случая.
\begin{theorem}[Неравенство Маркова]
	Пусть \(\xi\)~--- неотрицательная случайная величина. Тогда для любого \(\epsilon > 0\)
	\[
		\Pr{\xi \geq \epsilon} \leq \frac{\E{\xi}}{\epsilon}
	\]
\end{theorem}
\begin{proof}
	\[
		\Pr{\xi \geq \epsilon} = \E{\mathbf{1}_{\{\xi \geq \epsilon\}}} \leq \E{\frac{\xi}{\epsilon}\mathbf{1}_{\{\xi \geq \epsilon\}}} \leq \E{\frac{\xi}{\epsilon}} = \frac{\E{\xi}}{\epsilon}\qedhere
	\]
\end{proof}

Из неравенства Маркова легко получить неравенство Чебышёва:
\begin{theorem}[Неравенство Чебышёва]
	Пусть \(\xi\)~---  случайная величина. Тогда для любого \(\epsilon > 0\)
	\[
		\Pr{\left|\xi - \E{\xi}\right| \geq \epsilon} \leq \frac{\D{\xi}}{\epsilon^2}
	\]
\end{theorem}
\begin{proof}
	Просто воспользуемся неравенством Маркова:
	\[
		\Pr{\left|\xi - \E{\xi}\right| \geq \epsilon} = \Pr{\left(\xi - \E{\xi}\right)^2 \geq \epsilon^2} \leq \frac{\E{\left(\xi - \E{\xi}\right)^2}}{\epsilon^2} = \frac{\D{\xi}}{\epsilon^2} \qedhere
	\]
\end{proof}

Теперь докажем ещё одно неравенство:
\begin{theorem}[Неравенство Йенсена]
	Пусть \(\xi\)~--- случайная величина, а \(\phi\)~--- некоторая выпуклая вниз функция. Тогда \(\phi(\E{\xi}) \leq \E{\phi(\xi)}\).
\end{theorem}
\begin{proof}
	Как известно, для выпуклых вниз функций выполнено следующее свойство: для любого \(x_0 \in \R\) существует \(\lambda(x_0)\) такое, что для любого \(x \in \R\)
	\[
		\phi(x) \geq \phi(x_0) + \lambda(x_0)(x - x_0)
	\]
	
	По сути, это говорит, что график лежит выше касательной. Теперь подставим \(x = \xi\), \(x_0 = \E{\xi}\):
	\[
		\phi(\xi) \geq \phi(\E{\xi}) + \lambda(\E{\xi})(\xi - \E{\xi})
	\]
	
	Осталось взять матожидание обеих частей и заметить, что \(\E{\xi - \E{\xi}} = 0\):
	\[
		\E{\phi(\xi)} \geq \phi(\E{\xi}) \qedhere
	\]
\end{proof}

Если же функция \(\phi(x)\) выпукла вверх, то знак меняется на противоположный.

\begin{example}
	Если \(\xi \geq 0\), то \(\E{e^{\xi}} \geq e^{\E{\xi}}\). Далее, \(\E{\xi^3} \geq (\E{\xi})^3\).
\end{example}

\begin{remark}
	В данном доказательстве была опущена одна весьма важная деталь: мы не требовали, чтобы \(\phi\) была борелевской, а ведь надо было. Однако любая выпуклая функция является борелевской. Это следует из того, что любую выпуклую функцию можно представить, как супремум счётного набора линейных функций, то есть существуют последовательности \(\{a_n\}\), \(\{b_n\}\) такие, что \(\phi(x) = \sup(a_{n}x + b_{n})\). Так как супремум от счётного множества измерим, то и \(\phi(x)\) измерима, а, следовательно, является борелевской.
\end{remark}

\subsection{Сходимости случайных величин}

В дискретном случае мы уже вводили сходимости случайных величин. Почему бы и не обобщить этот результат? Рассмотрим некоторое вероятностное пространство \((\Omega, \F, \Pr)\) и \(\{\xi_n \mid n \in \N\}\)~--- последовательность случайных величин на этом пространстве. Далее, пусть \(\xi\)~--- это некоторая случайная величина.
\begin{definition}
	Будем говорить, что \(\xi_n\) сходится к \(\xi\) \emph{почти наверное} (с вероятностью 1), если
	\[
		\Pr{\left\{\omega \in \Omega : \lim\limits_{n \to \infty} \xi_n(\omega) = \xi(\omega)\right\}} = 1
	\]
	
	Обозначение: \(\xi_n \asto \xi\).
\end{definition}

\textbf{Смысл:} сходимость почти наверное~--- это поточечная сходимость.

\begin{definition}
	Будем говорить, что \(\xi_n\) сходится к \(\xi\) \emph{по вероятности}, если для любого \(\epsilon > 0\)
	\[
	\lim\limits_{n \to \infty} \Pr{|\xi_n - \xi| \geq \epsilon} = 0
	\]
	
	Обозначение: \(\xi_n \prto \xi\).
\end{definition}

\textbf{Смысл:} \(\xi_n\) мало отклоняются от \(\xi\).

\begin{definition}
	Будем говорить, что \(\xi_n\) сходится к \(\xi\) \emph{в среднем порядка \(p\)}, если
	\[
	\lim\limits_{n \to \infty} \E{|\xi_n - \xi|^p} = 0
	\]
	
	Обозначение: \(\xi_n \xrightarrow{L^p} \xi\).
\end{definition}

\textbf{Смысл:} сходимость последовательности в пространстве \(L^{p} = \{\xi : \E|\xi|^p < +\infty\}\).

\begin{definition}
	Будем говорить, что \(\xi_n\) сходится к \(\xi\) \emph{по распределению}, если для всех \(x\) точек непрерывности функции $F_{\xi}(x)$
	\[
	\lim\limits_{n \to \infty} F_{\xi_n}(x) = F_{\xi}(x)
	\]
	
	Обозначение: \(\xi_n \dto \xi\).
\end{definition}

\textbf{Смысл:} аппроксимация распределения.

Теперь напомню вам несколько результатов, связанных с сходимостью случайных величин. Для простоты введём следующее: если \(\xi_n \to \xi\), где \(\xi\) обладает каким-то распределением \(T\), то будем писать, что \(\xi_n \to T\).

\begin{theorem}[Пуассон]
	Пусть \(\xi_n \sim \mathrm{Bin}(n, p(n))\), где \(p(n) = \frac{\lambda}{n}\), \(\lambda > 0\). Тогда \(\xi_n \dto \mathrm{Pois}(\lambda)\). 
\end{theorem}
\begin{theorem}[Муавр, Лаплас]
	Пусть \(\xi_n \sim \mathrm{Bin}(n, p)\), где \(p \in (0, 1)\)~--- фиксированное число. Тогда
	\[
		\lim\limits_{n \to \infty} \Pr{\frac{\xi_n - \E{\xi}}{\sqrt{\D{\xi}}} \leq x} = \Phi(x) = \int\limits_{-\infty}^{x} \frac{1}{\sqrt{2\pi}}e^{-\frac{y^2}{2}}\diff y
	\]
	где \(\Phi(x)\)~--- это функция распределения \(\mathcal{N}(0, 1)\). По-другому это можно записать так:
	\[
		\frac{\xi_n - \E{\xi}}{\sqrt{\D{\xi}}} \dto \mathcal{N}(0, 1)
	\]
\end{theorem}
\begin{theorem}[закон больших чисел в форме Чебышёва]
	Пусть \(\{\xi_n \mid n \in \N\}\)~--- последовательность попарно независимых случайных величин таких, что все дисперсии конечны: то есть, существует константа \(C \in \R\) такая, что для любого натурального \(n\) \(\D{\xi_n} \leq C\). Теперь, введём обозначение \(S_n = \xi_1 + \dots + \xi_n\). Тогда
	\[
		\frac{S_n - \E{S_n}}{n} \prto 0
	\]
\end{theorem}

Теперь вернёмся к новому материалу.
\begin{definition}
	Пусть \(\{A_n \mid n \in \N\}\)~--- это счётная последовательность событий. Тогда событием \(\{A_n \text{ б.ч.}\}\) (\emph{событием, происходящим бесконечное число раз}) назовём событие, устроенное следующим образом:
	\[
		\omega \in \{A_n \text{ б.ч.}\} \iff \exists \{A_{n_k} \mid k \in \N\} : \forall k \in \N\ \omega \in A_{n_k}
	\]
	
	Формально это событие можно записать следующим образом:
	\[
		\{A_n \text{ б.ч.}\} = \bigcap_{n = 1}^{\infty}\bigcup_{n = m}^{\infty} A_{m}
	\]
\end{definition}

\begin{lemma}[Критерий сходимости почти наверное]
	Пусть \(\{\xi_n \mid n \in \N\}\), \(\xi\)~--- случайные величины. Тогда \(\xi_n \asto \xi\) тогда и только тогда, когда для любого \(\epsilon > 0\)
	\[
		\Pr{\{A_n^{\epsilon} \text{ б.ч.}\}} = 0,\text{ где } A_n^{\epsilon} = \{|\xi_n - \xi| \geq \epsilon\}
	\]
	Или же, что равносильно:
	\[
		\forall \epsilon > 0 \lim\limits_{n \to \infty} \Pr{\sup\limits_{m \geq n} |\xi_m - \xi| \geq \epsilon} = 0.
	\]
\end{lemma}
\begin{proof}
	Из определения предела последовательности следует, что \(\xi_n(\omega) \to \xi(\omega)\) равносильно тому, что для любого \(\epsilon > 0\) событий вида \(\{|\xi_n(\omega) - \xi(\omega)| \geq \epsilon\}\) лишь конечное число. Следовательно, \(\Pr{\xi_n \not\to \xi \text{ почти наверное}} = 0\) можно записать следующим образом:
	\[
		\Pr{\bigcup_{m = 1}^{\infty} \{A_n^{1/m} \text{ б.ч.}\}} = 0 \iff \forall m \in \N \Pr{\{A_n^{1/m} \text{ б.ч.}\}} = 0
	\]
	
	Так как для любого \(\epsilon > \frac{1}{m}\) \(A_n^{\epsilon} \subseteq A_n^{1/m}\), то это равносильно тому, что
	\[
		\forall \epsilon > 0 \Pr{\{A_n^{\epsilon} \text{ б.ч.}\}} = 0 \iff \forall \epsilon > 0 \Pr{\bigcap_{n = 1}^{\infty}\bigcup_{n = m}^{\infty} A_{m}^{\epsilon}} = 0
	\]
	
	Теперь введём события
	\[
		B_{n}^{\epsilon} = \bigcup_{m = n}^{\infty} A_{m}^{\epsilon}
	\]
	
	Несложно заметить, что \(B_{n}^{\epsilon} \supseteq B_{n + 1}^{\epsilon}\). Тогда по непрерывности верояностной меры
	\[
		\lim\limits_{n \to \infty} \Pr{B_{n}^{\epsilon}} = 0 = \lim\limits_{n \to \infty} \Pr{\bigcup_{m = n}^{\infty} A_{m}^{\epsilon}}
	\]
	
	Осталось заметить две вещи:
	\begin{enumerate}
		\item Так как \(\{\bigcup_{m = n}^{\infty} |\xi_m - \xi| \geq \epsilon\} \subseteq \{\sup_{m \geq n} |\xi_m - \xi| \geq \epsilon\}\), то
		\[
			\Pr{\sup_{m \geq n} |\xi_m - \xi| \geq \epsilon} = 0 \implies \Pr{\bigcup_{m = n}^{\infty} |\xi_m - \xi| \geq \epsilon} = 0
		\]
		
		\item Однако:
		\[
			\Pr{\bigcup_{m = n}^{\infty} |\xi_m - \xi| \geq \epsilon} = 0 \implies \Pr{\sup_{m \geq n} |\xi_m - \xi| \geq 2\epsilon} = 0
		\]
	\end{enumerate}
	Тем самым получаем желаемое.
\end{proof}

\begin{consequence}
	Если для любого \(\epsilon > 0\) ряд \(\sum \Pr{|\xi_n - \xi| \geq \epsilon}\) сходится, то \(\xi_n \asto \xi\).
\end{consequence}
\begin{proof}
	Как мы показывали, \(\xi_n \asto \xi\) равносильно тому, что для любого \(\epsilon > 0\) \(\lim\limits_{n \to \infty} \Pr{B_{n}^{\epsilon}} = 0\). Однако,
	\[
		\Pr{B_{n}^{\epsilon}} = \Pr{\bigcup_{m = n}^{\infty} |\xi_m - \xi| \geq \epsilon} \leq \sum_{m = n}^{\infty} \Pr{|\xi_m - \xi| \geq \epsilon} \xrightarrow{n \to \infty} 0\text{ (как остаток ряда)} \qedhere
	\]
\end{proof}

Казалось бы, зачем столько видов сходимостей? Они настолько разные? Оказывается, что да.
\begin{theorem}[о взаимоотношении сходимостей]
	Пусть \(\{\xi_n \mid n \in \N\}, \xi\)~--- случайные величины. Тогда
	\begin{enumerate}
		\item \(\xi_n \asto \xi \implies \xi_n \prto \xi\)
		\item \(\xi_n \xrightarrow{L^p} \xi \implies \xi_n \prto \xi\)
		\item \(\xi_n \prto \xi \implies \xi_n \dto \xi\)
	\end{enumerate}
	В обратную сторону утверждения в общем случае не верны. Про взаимосвязь сходимости почти наверное и в среднем ничего сказать нельзя.
\end{theorem}
\begin{proof}
	Сперва докажем, что из сходимости почти наверное следует сходимость по вероятности. Действительно, пусть \(\xi_n \asto \xi\). По доказанному выше критерию
	\[
		\forall \epsilon > 0 \lim\limits_{n \to \infty} \Pr{\sup\limits_{m \geq n} |\xi_m - \xi| \geq \epsilon} = 0.
	\]
	
	Однако
	\[
		\Pr{|\xi_n - \xi| \geq \epsilon} \leq \Pr{\sup\limits_{m \geq n} |\xi_m - \xi| \geq \epsilon} \xrightarrow{n \to \infty} 0
	\]
	
	Теперь покажем, что из сходимости в среднем порядка \(p\) следует сходимость по вероятности. На самом деле, по неравенству Маркова
	\[
		\Pr{|\xi_n - \xi| \geq \epsilon} = \Pr{|\xi_n - \xi|^p \geq \epsilon^p} \leq \frac{\E|\xi_n - \xi|^p}{\epsilon^p} \xrightarrow{n \to \infty} 0.
	\]
	
	Теперь приступим к самому сложному: доказательству того, что из сходимости по вероятности следует сходимость по распределению. Пусть \(\xi_n \prto \xi\) и \(x\)~--- это точка непрерывности \(F_{\xi}\). Тогда
	\begin{align*}
		F_{\xi_n}(x) &= \Pr{\xi_n \leq x} = \Pr{\xi_n \leq x, \xi - \xi_n \leq \epsilon} + \Pr{\xi_n \leq x, \xi - \xi_n > \epsilon} \\
		&\leq \Pr{\xi \leq x + \epsilon} + \Pr{|\xi - \xi_n| > \epsilon}
	\end{align*}
	
	Так как \(\xi_n \prto \xi\), то вторая вероятность стремится к \(0\) и 
	\[
		\varlimsup\limits_{n \to \infty} F_{\xi_n}(x) \leq F_{\xi}(x + \epsilon)
	\]
	
	Аналогичными рассуждениями получаем, что
	\[
		\varlimsup\limits_{n \to \infty} (1 - F_{\xi_n}(x)) = 1 - \varliminf\limits_{n \to \infty} F_{\xi_n}(x) \leq 1 - F_{\xi}(x - \epsilon) 
	\]
	
	Тогда
	\[
		F_{\xi}(x - \epsilon) \leq \varliminf\limits_{n \to \infty} F_{\xi_n}(x) \leq \varlimsup\limits_{n \to \infty} F_{\xi_n}(x) \leq F_{\xi}(x + \epsilon)
	\]
	
	Так как\(x\)~--- точка непрерывности, то, устремляя \(\epsilon\) к нулю, получаем, что
	\[
		\exists\lim\limits_{n \to \infty} F_{\xi_n}(x) = F_{\xi}(x)
	\]
	
	На тему остальных связей можно посмотреть книжку Стоянова "Контрпримеры в теории вероятностей".
\end{proof}

Однако в некторых случаях можно перейти от более слабой к более сильной сходимости.
\begin{point}
	Пусть \(\{\xi_n \mid n \in \N\}, \xi\)~--- случайные величины. Если \(\xi_n \prto \xi\), то существует подпоследовательность \(\{\xi_{n_k} \mid k \in \N\}\) такая, что \(\xi_{n_k} \asto \xi\).
\end{point}
\begin{proof}
	Для каждого \(k \in \N\) выберем \(n_{k}\) такое, что
	\[
		\forall n > n_{k} \Pr{|\xi_n - \xi| \geq \frac{1}{k}} \leq \frac{1}{2^k}
	\]
	
	Но тогда
	\begin{align*}
		\sum_{k = 1}^{\infty} \Pr{|\xi_{n_k} - \xi| \geq \epsilon} &= \left|K > \frac{1}{\epsilon}\right| = O\left(\sum_{k = K}^{\infty} \Pr{|\xi_{n_k} - \xi| \geq \epsilon}\right) \\
		&= O\left(\sum_{k = K}^{\infty} \Pr{|\xi_{n_k} - \xi| \geq \frac{1}{k}}\right) < +\infty
	\end{align*}
	
	Это и означает, что \(\xi_{n_k} \asto \xi\).
\end{proof}