\section{Лекция от 10.02.2017}
\subsection{Вычисление условного математического ожидания}
Ранее мы ввели условное математическое ожидание. Теперь хотелось бы научиться считать его. Для начала покажем, как это делается в дискретном случае.
\begin{theorem}
	Пусть \(\eta\) имеет дискретное распределение и \(X = (x_1, \dots, x_n, 
	\dots)\)~--- множество значений. Тогда для любой случайной величины \(\xi\) с конечным матожиданием
	выполнено, что
	\[
		\E{\xi \given \eta} = \sum_{k = 1}^{\infty} 
		\frac{\E{\xi\mathbf{1}_{\{\eta = x_{k}\}}}}{\Pr{\eta = x_{k}}} 
		\mathbf{1}_{\{\eta = x_{k}\}}
	\]
\end{theorem}
\begin{proof}
	Достаточно проверить измеримость и интегральное свойство. Как известно, \(\mathbf{1}_{\{\eta = x\}}\) является борелевской функцией от \(\eta\) для любого \(x\). Тогда функция в правой части является борелевской функцией от \(\eta\).
	
	Теперь проверим интегральное свойство. Для начала возьмём \(B = \{x_n\}\), где \(x_n\)~--- это одно из значений \(\eta\), для которого \(\Pr{\eta = x_n} > 0\).
	\begin{align*}
		\E{\E{\xi \given \eta}\mathbf{1}_{\{\eta = x_{n}\}}} &= \E{\frac{\E{\xi 
		\mathbf{1}_{\{\eta = x_{n}\}}}}{\Pr{\eta = x_n}} \mathbf{1}_{\{\eta = 
		x_{n}\}}} \\
		&= \frac{\E{\xi\mathbf{1}_{\{\eta = x_{n}\}}}}{\Pr{\eta = 
		x_n}}\E{\mathbf{1}_{\{\eta = x_{k}\}}} = \E{\xi\mathbf{1}_{\{\eta = 
		x_{k}\}}}
	\end{align*}
	Теперь обобщим это на произвольное \(B\). Пусть в \(B\) есть элементы 
	\(y_{1}, \dots, y_{n}, \dots\), которые являются значениям \(\eta\). Тогда 
	по линейности матожидания получаем, что
	\begin{align*}
		\E{\xi\mathbf{1}_{B}(\eta)} &= \E{\xi\left(\sum_{i = 1}^{\infty} 
		\mathbf{1}_{\{\eta = y_{i}\}}\right)} = \sum_{i = 1}^{\infty} \E{\xi 
		\mathbf{1}_{\{\eta = y_{i}\}}} \\
		&= \lim\limits_{n \to \infty} \sum_{i = 1}^{n} \E{\xi 
		\mathbf{1}_{\{\eta = y_{i}\}}}
	\end{align*}
	Заметим, что \(|\xi \mathbf{1}_{\{\eta = y_{i}\}}| \leq |\xi|\) и \(\E{\xi 
	\mathbf{1}_{\{\eta = y_{i}\}}} = \E{\E{\xi \given \eta}\mathbf{1}_{\{\eta = 
	y_{i}\}}}\). Тогда по теореме Лебега о мажорируемой сходимости
	\[
		\lim\limits_{n \to \infty} \sum_{i = 1}^{n} \E{\xi \mathbf{1}_{\{\eta = 
		y_{i}\}}} = \E{\E{\xi \given \eta} \left(\sum_{i = 1}^{\infty} 
		\mathbf{1}_{\{\eta = y_{i}\}}\right)} = \E{\E{\xi \given \eta} 
		\mathbf{1}_{B}(\eta)}.\qedhere
	\]
\end{proof}

\subsection{Свойства условного математического ожидания}
У условного математического ожидания немало полезных свойств.
\begin{property}
	Если \(\xi = f(\eta)\), где \(f\)~--- некоторая борелевская функция, то \(\E{\xi \given \eta} = \xi\).
\end{property}
\begin{proof}
	Несложно проверить, что для \(\xi\) выполняются свойство измеримости и интегральное свойство.
\end{proof}
\begin{property}[Линейность]
	Для любых случайных величин (или векторов) \(\xi\), \(\eta\) и \(\delta\) выполнено, что
	\[
		\forall a, b \in \R\ \E{a\xi + b\delta \given \eta} = a\E{\xi \given \eta} + b\E{\delta \given \eta}.
	\]
\end{property}
\begin{proof}
	Для начала заметим, что и \(\E{\xi \given \eta}\), и \(\E{\delta \given \eta}\) являются борелевскими функциями от \(\eta\). Тогда \(a\E{\xi \given \eta} + b\E{\delta \given \eta}\) тоже есть борелевская функция от \(\eta\).
	
	Осталось проверить интегральное свойство. По линейности матожидания
	\[
		\E{(a\xi + b\delta)\mathbf{1}_{B}(\eta)} = a\E{\xi\mathbf{1}_{B}(\eta)} + b\E{\delta\mathbf{1}_{B}(\eta)}.
	\]
	Согласно интегральному свойству:
	\[
		a\E{\xi\mathbf{1}_{B}(\eta)} + b\E{\delta\mathbf{1}_{B}(\eta)} = a\E{\E{\xi \given \eta}\mathbf{1}_{B}(\eta)} + b\E{\E{\delta \given \eta}\mathbf{1}_{B}(\eta)}.
	\]
	Свернём сумму назад:
	\[
		a\E{\E{\xi \given \eta}\mathbf{1}_{B}(\eta)} + b\E{\E{\delta \given \eta}\mathbf{1}_{B}(\eta)} = \E{(a\E{\xi \given \eta} + b\E{\delta \given \eta})\mathbf{1}_{B}(\eta)}.
	\]
	Тем самым получаем желаемое.
\end{proof}

\begin{property}[Формула полной вероятности]
	<<Математическое ожидание убирает условие>>: \(\E{\E{\xi \given \eta}} = \E{\xi}\).
\end{property}
\begin{proof}
	Пусть \(\eta \in \R^n\). Воспользуемся интегральным свойством, положив \(B = \R^n\):
	\[
		\E{\xi} = \E{\xi\mathbf{1}_{B}(\eta)} = \E{\E{\xi \given \eta}\mathbf{1}_{B}(\eta)} = \E{\E{\xi \given \eta}}.\qedhere
	\]
\end{proof}

\begin{property}
	Если \(\xi\) независимо с \(\eta\), то \(\E{\xi \given \eta} = \E{\xi}\).
\end{property}
\begin{proof}
	Заметим, что \(\E{\xi}\)~--- константа. Тогда можно считать, что \(\E{\xi}\) есть борелевская функция от \(\eta\). Проверим интегральное свойство:
	\[
		\E{\xi\mathbf{1}_{B}(\eta)} = \left|\text{по независимости}\right| = \E{\xi}\E{\mathbf{1}_{B}(\eta)} = \E{\E{\xi}\mathbf{1}_{B}(\eta)}.\qedhere
	\]
\end{proof}

\begin{property}[Сохранение отношения порядка]
	Если \(\xi \leq \delta\) почти наверное, то \(\E{\xi \given \eta} \leq \E{\delta \given \eta}\) почти наверное.
\end{property}
\begin{proof}
	Рассмотрим случайную величину \(\chi = \E{\xi \given \eta} - \E{\delta \given \eta}\). Несложно понять, что она является борелевской функцией от \(\eta\). Теперь рассмотрим \(\E{\chi\mathbf{1}_{(0, +\infty)}(\chi)}\). Заметим, что \(\E{\chi\mathbf{1}_{(0, +\infty)}(\chi)} \geq 0\). Теперь скажем, что \(\mathbf{1}_{(0, +\infty)}(\chi) = \mathbf{1}_{B}(\eta)\) для какого-то борелевского множества \(B\).\footnote{На самом деле \(B = \chi^{-1}(0, +\infty)\).} Тогда
	\[
		\E{\chi\mathbf{1}_{(0, +\infty)}(\chi)} = \E{(\E{\xi \given \eta} - \E{\delta \given \eta})\mathbf{1}_{B}(\eta)}.
	\]
	Пользуясь линейностью матожидания и интегральным свойством, получаем, что
	\[
		\E{\chi\mathbf{1}_{(0, +\infty)}(\chi)} =  \E{\xi\mathbf{1}_{B}(\eta)} - \E{\delta \mathbf{1}_{B}(\eta)} \leq 0 \text{, так как } \xi \leq \delta.
	\]
	Тогда получаем, что \(\chi\mathbf{1}_{(0, +\infty)}(\chi) = 0\) почти наверное и \(\xi \leq \delta\).
\end{proof}

\begin{property}[Телескопическое]
	Пусть \(\xi\), \(\eta\) и \(\delta\)~--- случайные векторы. Тогда
	\begin{enumerate}
		\item \(\E{\E{\xi \given \eta} \given (\eta, \delta)} = \E{\xi \given \eta}\),
		\item \(\E{\E{\xi \given (\eta, \delta)} \given \eta} = \E{\xi \given \eta}\).
	\end{enumerate}
	
\end{property}
\begin{proof}
	Первый пункт почти очевиден. Заметим, что \(\E{\xi \given \eta}\)~--- борелевская функция от \(\eta\). Но тогда она является борелевской функцией от \((\eta, \delta)\). Следовательно, по свойству 1 получаем желаемое.
	
	Теперь докажем второе утверждение. Заметим, что \(\E{\xi \given \eta}\) является борелевской функцией от \(\eta\). Теперь проверим интегральное свойство. Пусть \(\delta \in \R^{m}\). Тогда по интегральному свойству:
	\begin{align*}
		\E{\E{\xi \given (\eta, \delta)}\mathbf{1}_{B}(\eta)} &= \E{\E{\xi \given (\eta, \delta)}\mathbf{1}_{B \times \R^{m}}(\eta, \delta)} = \\
		&= \E{\xi\mathbf{1}_{B \times \R^{m}}(\eta, \delta)} = \E{\xi\mathbf{1}_{B}(\eta)} = \E{\E{\xi \given \eta}\mathbf{1}_{B}(\eta)}.\qedhere
	\end{align*}
\end{proof}

\begin{property}[Предельный переход]
	Пусть \(\{\xi_n \mid n \in \N\}\), \(\xi\), \(\eta\), \(\delta\)~--- случайные величины, где $\E{|\delta|} < \infty, 
	\E{|\xi|} < \infty, \E{|\xi_n|} < \infty$.
	\begin{enumerate}
		\item Если \(0 \leq \xi_n \uparrow \xi\) почти наверное, то \(\E{\xi_n \given \eta} \uparrow \E{\xi \given \eta}\) почти наверное.
		\item Если \(\xi_n \asto \xi\) и \(|\xi_n| \leq |\delta|\) для всех натуральных \(n\), то \(\E{\xi_n \given \eta} \asto \E{\xi \given \eta}\).
	\end{enumerate}
\end{property}
\begin{proof}
	Так как \(\xi_n\) монотонно возрастают, то и \(\E{\xi_n \given \eta}\) монотонно возрастают (по свойству 5). Тогда существует (не обязательно конечный) предел \(h = \lim\limits_{n \to \infty}\E{\xi_n \given \eta}\). Так как для любого натурального \(n\) \(\E{\xi_n \given \eta}\)~--- это борелевская функция от \(\eta\), то и \(h\) является борелевской функцией от \(\eta\).
	
	Теперь возьмём произвольное борелевское множество \(B\) и воспользуемся интегральным свойством:
	\[
		\E{\xi_n \mathbf{1}_{B}(\eta)} = \E{\E{\xi_n \given \eta}\mathbf{1}_{B}(\eta)}.
	\]
	Теперь заметим, что \(\xi_n \mathbf{1}_{B}(\eta) \uparrow \xi \mathbf{1}_{B}(\eta)\) и \(\E{\xi_n \given \eta}\mathbf{1}_{B}(\eta) \uparrow h(\eta)\mathbf{1}_{B}(\eta)\). Тогда по теореме о монотонной сходимости
	\[
	\E{\xi\mathbf{1}_{B}(\eta)} = \E{h(\eta)\mathbf{1}_{B}(\eta)}.
	\]
	А это и означает, что \(\E{\xi_n \given \eta} \uparrow \E{\xi \given \eta}\).
	
	Второй же пункт доказывается по аналогии с доказательством теоремы Лебега для обычного математического ожидания.
\end{proof}

\begin{property}
	Если \(\delta\)~--- это борелевская функция от \(\eta\), то \(\E{\xi\delta \given \eta} = \delta\E{\xi \given \eta}\).
\end{property}
\begin{proof}
	Пусть \(\delta = \mathbf{1}_{A}(\eta)\), где \(A\)~--- некоторое борелевское множество. Заметим, что \(\delta\E{\xi \given \eta}\)~--- это борелевская функция от \(\eta\). Проверим интегральное свойство:
	\begin{align*}
		\E{\delta\xi\mathbf{1}_{B}(\eta)} &= \E{\xi\mathbf{1}_{A}(\eta)\mathbf{1}_{B}(\eta)} = \E{\xi\mathbf{1}_{A \cap B}(\eta)} = \E{\E{\xi \given \eta}\mathbf{1}_{A \cap B}(\eta)} \\
		&= \E{\E{\xi \given \eta}\mathbf{1}_{A}(\eta)\mathbf{1}_{B}(\eta)} = \E{\delta\E{\xi \given \eta}\mathbf{1}_{B}(\eta)}.
	\end{align*}
	Пользуясь линейностью матожидания, интегральное свойство будет выполнено и для простых функций вида \(\delta = \sum c_k\mathbf{1}_{A_k}(\eta)\).
	
	В общем случае для любой \(\delta = f(\eta)\) построим последовательность простых функций \(\delta_n = f_{n}(\eta)\) такую, что \(\delta_n \asto \delta\) и \(|\delta_n| \leq |\delta|\). Тогда по свойству 7 получаем, что \(\E{\delta_n\xi \given \eta} \asto \E{\delta\xi \given \eta}\). Однако
	\[
		\E{\delta_n\xi \given \eta} = \delta_n\E{\xi \given \eta} \asto \delta\E{\xi \given \eta}.
	\]
	Следовательно, \(\E{\xi\delta \given \eta} = \delta\E{\xi \given \eta}\) (почти наверное, конечно).
\end{proof}

\subsection{Условные распределения и плотности}
\begin{definition}
	\emph{Условным математическим ожиданием при условии \(\eta = y\)}, \(\eta, y \in \R^n\) называется \(\E{\xi \given \eta = y}\).
\end{definition}
Вообще говоря, \(\E{\xi \given \eta = y}\)~--- это борелевская функция \(\phi(y)\) такая, что для любого \(B \in \B(\R^n)\)
\[
	\E{\xi\mathbf{1}_{B}(\eta)} = \E{\phi(\eta)\mathbf{1}_{B}(\eta)}.	
\]
Отсюда можно сделать простой вывод: \(\E{\xi \given \eta = y} = \phi(y) \iff \E{\xi \given \eta} = \phi(\eta)\).

Свойства у такого условного математического ожидания похожи на свойства для обычного условного: например, выполнены линейность, сохранение относительного порядка и теорема Лебега.

Впрочем, мы не имеем представления о том, как его вычислять. Для этого введём условные распределение и плотность.
\begin{definition}
	\emph{Условным распределением} случайной величины \(\xi\) при условии, что \(\eta = y\) назовём функцию \(\Pr{\xi \in B \given \eta = y} \equiv \E{\mathbf{1}_{B}(\xi) \given \eta = y}\), рассматриваемую, как функцию от \(B \in \B(\R^n)\) при фиксированном \(y \in \R^k\).
\end{definition}

\begin{point}
	При фиксированном \(y \in \R^k\) с вероятностью 1 (по распределению \(\eta\)) условное распределение есть распределение вероятностей на \((\R, \B(\R))\).
\end{point}

\begin{definition}
	Если условное распределение имеет плотность \(p_{\xi \mid \eta}(x \mid y)\), то назовём его \emph{условной плотностью \(\xi\) относительно \(\eta\)}. То есть для любого \(B \in \B(\R^n)\)
	\[
		\Pr{\xi \in B \given \eta = y} = \int\limits_{B} p_{\xi \mid \eta}(x \mid y)\diff x.
	\]
\end{definition}

Теперь сформулируем две теоремы (пока что без доказательства), которые и позволяют нам считать условное математическое ожидание.
\begin{theorem}[о вычислении условного математического ожидания]
	Пусть \(\xi\)~--- случайная величина, а \(f(x)\)~--- некоторая борелевская функция. Если \(\E{|f(\xi)|} < +\infty\) и существует плотность \(p_{\xi \mid \eta}(x \mid y)\), то
	\[
		\E{f(\xi) \given \eta = y} = \int\limits_{\R} f(x)p_{\xi \mid \eta}(x \mid y)\diff x.
	\]
\end{theorem}
\begin{theorem}
	Пусть \(\xi\) и \(\eta\) таковы, что есть совместная плотность \(p_{\xi, \eta}(x, y)\). Тогда существует условная плотность \(p_{\xi \mid \eta}(x \mid y)\) и она равна
	\[
		p_{\xi \mid \eta}(x \mid y) = \begin{cases}
		\dfrac{p_{\xi, \eta}(x, y)}{p_{\eta}(y)},& p_{\eta}(y) > 0, \\
		0,& p_{\eta}(y) = 0.
		\end{cases}
	\]
\end{theorem}

Рассмотрим на примере, как их использовать.
\begin{problem}
	Пусть \(\xi\) и \(\eta\)~--- независимые случайные величины, причём \(\xi, \eta \sim \mathrm{Exp}(1)\). Найти \(\E{\xi^2 \given \xi + \eta}\).
\end{problem}
\begin{proof}[Решение]
	Для начала найдём плотность \(\xi + \eta\). По формуле свёртки:
	\begin{align*}
		p_{\xi + \eta}(x) &= \int_{-\infty}^{+\infty} e^{-(x - y)}e^{-y}\mathbf{1}_{(0, +\infty)}(x - y)\mathbf{1}_{(0, +\infty)}(y)\diff y \\
		&= \mathbf{1}_{(0, +\infty)}(x) \int_{0}^{x} e^{-x}\diff y = xe^{-x}\mathbf{1}_{(0, +\infty)}(x).
	\end{align*}
	
	Теперь найдём совместную плотность распределения \(\xi\) и \(\xi + \eta\). Для этого распишем вероятностную меру:
	\[
		\Pr{(\xi, \xi + \eta) \in B} = \iint\limits_{(x, x + y) \in B} e^{-x}e^{-y}\mathbf{1}_{(0, +\infty)}(x)\mathbf{1}_{(0, +\infty)}(y)\diff x \diff y
	\]
	Сделаем замену переменной:
	\[
		\begin{cases}
			u = x \\
			v = x + y
		\end{cases}
		\implies
		\begin{cases}
			x = u \\
			y = u - v
		\end{cases}
		\implies
		\mathbf{J}(u, v) = 
		\begin{vmatrix}
		1 & 0 \\
		1 & -1
		\end{vmatrix}
	\]
	Тогда \(|\det\mathbf{J}(u, v)| = 1\) и
	\[
		\Pr{(\xi, \xi + \eta) \in B} = \iint\limits_{(u, v) \in B} e^{-v}\mathbf{1}_{\{0 < u < v\}}\diff u \diff v.
	\]
	Следовательно, \(p_{\xi, \xi + \eta}(x, y) = e^{-y}\mathbf{1}_{\{0 < x < y\}}\). Теперь посчитаем \(p_{\xi \mid \eta}(x \mid y)\). Так как \(\xi + \eta > 0\) с вероятностью 1, то:
	\[
		p_{\xi \mid \eta}(x \mid y) = \dfrac{e^{-y}\mathbf{1}_{\{0 < x < y\}}}{ye^{-y}\mathbf{1}_{(0, +\infty)}(y)} = \frac{1}{y}\mathbf{1}_{\{0 < x < y\}}.
	\]
	Недолго думая, считаем условное матожидание по формуле:
	\[
		\E{\xi^2 \given \xi + \eta = y} = \int_{0}^{y}\frac{x^2\diff x}{y} = \frac{y^2}{3} \implies \E{\xi^2 \given \xi + \eta} = \frac{(\xi + \eta)^2}{3}.
	\]
\end{proof}
