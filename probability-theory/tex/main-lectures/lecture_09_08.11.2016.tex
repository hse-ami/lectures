\section{Лекция от 08.11.2016}

\subsection{Канторова лестница. Продолжение.}
Заметим, что канторова лестница~--- неубывающая непрерывная функция, которая возрастает от 0 до 1. Давайте рассмотрим множество её точек роста:

Легко понять, что точками роста могут быть только точки, не попавшие ни в один заполняемый интервал. Давайте оценим меру Лебега множества точек роста, пользуясь аддитивностью меры и тем, что мера $[0, 1]$ равна 1.

Сумма длин интервалов на $i$-ом шаге --- $\frac{1}{3}\cdot\left(\frac{2}{3}\right)^i$; тогда всего
эти интервалы заполняют множество меры
$\sum\limits_{i = 0}^{\infty}\frac{1}{3}\cdot\left(\frac{2}{3}\right)^i = 1$. Значит, мера множества точек
роста равна $1-1 = 0$, значит, канторова лестница является сингулярной функцией распределения.

Казалось бы, а зачем вообще нужны сингулярные функции? Оказывается, что дискретные, абсолютно непрерывные и синуглярные функции покрывают все функции распределения.

\begin{theorem}[Лебег]
    Пусть $F(x)$ --- функция распределения на $\R$. Тогда имеет место представление
    $F(x) = \alpha_1 F_1(x) + \alpha_2 F_2(x) + \alpha_3 F_3(x)$, где
    \begin{itemize}
        \item $F_1$ --- дискретная;
        \item $F_2$ --- абсолютно непрерывная;
        \item $F_3$ --- сингулярная.
    \end{itemize}

    При этом $a_i \geq 0,\ \alpha_1 + \alpha_2 + \alpha_3 = 1$.
\end{theorem}

\begin{proof}
    Это хорошая теорема. Доказывать её мы, конечно, не будем.
\end{proof}

\subsection{Случайные величины и векторы}

В дискретном случае мы говорили, что \(\xi\)~--- это произвольное отображение из \(\Omega\) в \(R\). Но в дискретном случае никаких сложностей не возникает, так как всегда можно посчитать любую вероятность. В общем случае такой красоты нет. Тогда нужно вводить некоторые ограничения.

Однако, было бы неплохо ввести обобщение так, чтобы дискретный случай был частным случаем общего (как и должно быть). То есть, если на вероятностном пространстве \((\Omega, \F, \Pr)\) задана некоторая случайная величина \(\xi\), то мы хотим уметь считать вероятности вида \(\Pr{\xi \leq x}\), \(\Pr{\xi = x}\), \(\Pr{a \leq \xi \leq b}\) и так далее. Поэтому придётся ввести одно ограничение. 

\begin{definition}
	Пусть задано вероятностное пространство \((\Omega, \F, \Pr)\). Будем называть \emph{случайной величиной} отображение \(\xi : \Omega \mapsto \R\), если оно удовлетворяет свойству \emph{измеримости}: для любого \(x \in \R\) событие \(\{\xi \leq x\} \equiv \{\omega \in \Omega \mid \xi(\omega) \leq x\}\) принадлежит \(\F\).
\end{definition}

Смысл измеримости описан ранее и скрыт в названии: если функция измерима, то мы можем её измерить.

Если выбрать случайную точку на той же плоскости, то как её описать? Введём понятие \emph{случайного вектора}:

\begin{definition}
    \emph{Случайный вектор}~--- вектор, состоящий из случайных величин.
\end{definition}

Казалось бы, почему мы вводим только такие события? Оказывается, этого более, чем достаточно и для случайных величин, и для случайных векторов. Однако для того, чтобы доказать это, нужно ввести ещё одно определение:

\begin{definition}
    Борелевской $\sigma$-алгеброй в $\R^n$ называют минимальную 
    $\sigma$-алгебру содержащую все \((a_1,  b_1] \times (a_2, b_2] \times 
    \dots \times (a_n, b_n]\):
    \[
    \B(\R^n) = \sigma\left\{(a_1, b_1] \times \dots \times (a_n, b_n] \mid a_1, \dots, a_n, b_1, \dots, b_n \in \R, a_i < b_i\right\}.
    \]
\end{definition}

\begin{remark}
	Можно показать, что в \(\B(\R^n)\) лежит много чего: все прямоугольники вида \([a_1,  b_1] \times [a_2, b_2] \times \dots \times [a_n, b_n]\), все открытые и замкнутые множества, элементы вида \((-\infty,  b_1] \times (-\infty, b_2] \times \dots \times (-\infty, b_n]\) и так далее.
\end{remark}
Теперь докажем то, что нашего определения достаточно:
\begin{lemma}
	Пусть \((\Omega, \F, \Pr)\)~--- вероятностное пространство, и на нём заданы случайная величина \(\xi\) и случайный вектор \(\eta = (\eta_1, \dots, \eta_n)\). Тогда
	\begin{enumerate}
		\item \(\xi\) является случайной величиной тогда и только тогда, когда для любого \(B \in \B(\R)\) событие \(\xi^{-1}(B) = \{\xi \in B\}\) лежит в \(\F\).
		\item \(\eta\) является случайным вектором тогда и только тогда, когда 
		для любого \(B \in \B(\R^n)\) событие \(\eta^{-1}(B) = \{\eta \in B\}\) 
		лежит в \(\F\).
	\end{enumerate}
\end{lemma}
\begin{remark}
	Если \(B \in \B(\R^n)\), то его ещё (неформально) называют \emph{борелевским множеством}.
\end{remark}

\begin{proof}
	Для начала докажем это для случайных величин.
	\begin{itemize}
		\item[{\([\Leftarrow]\)}] Заметим, что \((-\infty, x] \in \B(\R)\) для любого \(x \in \R\). Тогда 
		\[
		\xi^{-1}\left((-\infty, x]\right) = \{\xi \in (-\infty, x]\} \in \F.
		\]
		А это и есть определение измеримости. Следовательно, \(\xi\)~--- случайная величина.
		
		\item[{\([\Rightarrow]\)}] Рассмотрим следующее множество:
		\[
		D = \left\{B \in \B(\R) \mid \xi^{-1}(B) \in \F\right\}.
		\]
		Покажем, что оно является \(\sigma\)-алгеброй. Так как полный прообраз сохраняет все теоретико-множественные операции, то в \(D\) будут находиться пересечения, дополнения, объединения и счётные объединения. Тогда \(D\) действительно является \(\sigma\)-алгеброй.
		
		Однако, согласно определению случайной величины, все лучи вида \((-\infty, x]\) лежат в \(D\). Тогда \(D\) есть \(\sigma\)-алгебра, содержащая все лучи. Но тогда она содержит в себе \(\B(\R)\). Так как по построению \(D \subseteq \B(\R)\), то \(D = \B(\R)\).
	\end{itemize}
	Идейно доказательство для случайных векторов ничем не отличается.
	\begin{itemize}
		\item[{\([\Leftarrow]\)}] Заметим, что \((-\infty, x_1] \times \dots \times (-\infty, x_n] \in \B(\R^n)\) для любых \(x_1, x_2, \dots, x_n \in \R\). Тогда 
		\[
		\eta^{-1}\left((-\infty, x_1] \times \dots \times (-\infty, x_n]\right) = \{\eta_1 \in (-\infty, x_1]\} \cap \dots \cap \{\eta_n \in (-\infty, x_n]\} \in \F.
		\]
		Тогда все \(\eta_1, \dots, \eta_n\) являются случайными величинами и \(\eta\) действительно является случайным вектором.
		
		\item[{\([\Rightarrow]\)}] Рассмотрим следующее множество:
		\[
		D = \left\{B \in \B(\R) \mid \eta^{-1}(B) \in \F\right\}.
		\]
		Покажем, что оно является \(\sigma\)-алгеброй. Так как полный прообраз сохраняет все теоретико-множественные операции, то в \(D\) будут находиться пересечения, дополнения, объединения и счётные объединения. Тогда \(D\) действительно является \(\sigma\)-алгеброй.
		
		Однако, согласно определению случайной величины, все элементы вида \((-\infty, x_1] \times \dots \times (-\infty, x_n]\) лежат в \(D\). Тогда \(D\) содержит в себе \(\B(\R)\). Так как по построению \(D \subseteq \B(\R)\), то \(D = \B(\R)\).
	\end{itemize}
\end{proof}

\subsection{Действия над случайными величинами}
Допустим, у нас есть несколько случайных величин \(\xi_1, \xi_2, \dots, \xi_n\), и мы хотим с ними что-то сделать~--- перемножить, например. Другими словами, мы хотим взять \emph{функцию} от случайных величин \(f(\xi_1, \xi_2, \dots, \xi_n)\). Но можно ли сказать, что это действительно будет случайной величиной? Не всегда. Однако есть класс функций, называемых \emph{борелевскими}, для которых это точно верно. Введём определение:
\begin{definition}
	Пусть \(f : \R^n \mapsto \R^m\)~--- некоторая функция. Будем называть её \emph{борелевской}, если выполнено следующее условие:
	\[
	\forall B \in \B(\R^m)\ f^{-1}(B) \equiv \{x : f(x) \in B\} \in \B(\R^n).
	\]
\end{definition}

Вообще, почти все ``обычные'' функции являются борелевскими. Например, непрерывные функции являются борелевскими, так как они переводят открытые множества в открытые. Можно так же доказать, что если у функции множество точек разрыва имеет лебегову меру 0, то она тоже является борелевской.\footnote{Доказательство можно найти по следующей ссылке: \url{http://math.stackexchange.com/questions/369268/a-function-with-countable-discontinuities-is-borel-measurable}}

Теперь покажем, что борелевские функции действительно ничего не ломают.
\begin{theorem}
	Пусть \(\xi_1, \ldots, \xi_n\) --- случайные величины, а \(f : \R^n \mapsto \R^m\)~--- борелевская функция. Тогда \(f(\xi_1, \xi_2, \dots, \xi_n)\) является случайной величиной.
\end{theorem}
\begin{proof}
	Рассмотрим произвольное борелевское множество \(B \in \B(\R^m)\). Тогда
	\[
	\{f(\xi_1, \dots, \xi_n) \in B\} = \left\{(\xi_1, \xi_2, \dots, \xi_n) \in f^{-1}(B)\right\}.
	\]
	Так как \(f^{-1}(B) \in \B(\R^{n})\), то \(\{f(\xi_1, \dots, \xi_n) \in B\} \in \F\).
\end{proof}
\begin{consequence}
	Если \(\xi\) и \(\eta\)~--- это случайные величины, то \(\xi \pm \eta\), \(\alpha\xi\) для любого \(\alpha \in \R\), \(\xi\eta\) и \(\xi/\eta\) при \(\eta \neq 0\) тоже являются случайными величинами.
\end{consequence}

В дальнейшем у нас будут возникать последовательности случайных величин. Хотелось бы понять~--- а будут ли те же пределы последовательностей случайными величинами? Ответ~--- да.
\begin{theorem}
	Пусть \(\{\xi_n\}_{n = 1}^{\infty}\)~--- последовательность случайных величин. Тогда 
	\[
	\varlimsup\limits_{n \to \infty} \xi_n, \varliminf\limits_{n \to \infty} \xi_n, \sup\limits_{n \in \N} \xi_n, \inf\limits_{n \in \N} \xi_n
	\]
	тоже являются случайными величинами.
\end{theorem}
\begin{proof}
	Покажем, что все они измеримы:
	\begin{align}
		\left\{\sup\limits_{n \in \N} \xi_n > x\right\} &= \bigcup\limits_{n = 1}^{\infty} \underbrace{\left\{\xi_n > x\right\}}_{\in \F} \in \F \\
		\left\{\inf\limits_{n \in \N} \xi_n < x\right\} &= \bigcup\limits_{n = 1}^{\infty} \underbrace{\left\{\xi_n < x\right\}}_{\in \F} \in \F \\
		\varlimsup\limits_{n \in \N} \xi_n &= \inf\limits_{n \in \N} \sup\limits_{k \geq n} \xi_k \\
		\varliminf\limits_{n \in \N} \xi_n &= \sup\limits_{n \in \N} \inf\limits_{k \geq n} \xi_k\qedhere
	\end{align}
\end{proof}

\subsection{Простые случайные величины. Матожидание в общем случае}

Напомним определение индикатора:
\begin{definition}
	Пусть \((\Omega, \F, \Pr)\)~--- вероятностное пространство и \(A \in \F\)~--- некоторое событие. Тогда \emph{индикатором} события \(A\) будем называть функцию \(\mathbf{I}_{A} : \Omega \mapsto \{0, 1\}\), устроенную следующим образом:
	\[
	\mathbf{I}_{A}(\omega) = \begin{cases}
	1,& \omega \in A \\
	0,& \omega \not\in A
	\end{cases}
	\]
	
	Иногда индикатор обозначают через \(\mathbf{I}\left\{A\right\}\).
\end{definition}

Покажем, что индикатор является случайной величиной. Действительно,
\[
\{\mathbf{I}_{A} \leq x\} = \begin{cases}
\emptyset,& x < 0 \\
\overline{A},& 0 \leq x < 1 \\
\Omega,& x \geq 1
\end{cases}
\]
Так как каждое из трёх подмножеств лежит в \(\F\), то \(\mathbf{I}_{A}\) является случайной величиной, что и требовалось доказать.

Индикатор является один из примеров так называемых \emph{простых} случайных величин. Введём определение:

\begin{definition}
    \emph{Простая случайная величина}~--- случайная величина, принимающая конечное число значений.
\end{definition}

Если \(x_1, x_2, \dots, x_n\)~--- все значения простой случайной величины \(\xi\), то её можно представить в следующем виде:
\[
\xi = \sum_{k = 1}^{n} x_k\mathbf{I}\{\xi = x_k\}
\]

Теперь о самом матожидании:

\begin{itemize}
	\item Пусть \(\xi\)~--- простая случайная величина. Тогда, по аналогии с дискретным случаем, \(\E{\xi}\) будет равно
	\[
	\E{\xi} \equiv \sum_{k = 1}^{n} x_k\Pr{\xi = x_k}.
	\]
	В дискретном случае похожая формула доказывалась, однако в общем случае она постулируется.
	
	\item Теперь допустим, что \(\xi\)~--- это произвольная неотрицательная случайная величина. Как ввести её матожидание? Допустим, что есть последовательность \(\{\xi_n\}_{n = 1}^{\infty}\) простых случайных величин такая, что \(\xi_n(\omega) \uparrow \xi(\omega)\) для любого \(\omega \in \Omega\). Тогда положим, что
	\[
	\E{\xi} \equiv \lim\limits_{n \to \infty} \E{\xi_n}.
	\]
	Однако возникает вопрос: а существует ли она? Докажем, что такая последовательность \(\xi_n\) действительно существует.
	
	Рассмотрим последовательность случайных величин \(\{\xi_n\}_{n = 1}^{\infty}\), устроенную следующим образом:
	\[
	\xi_n = \sum_{k = 1}^{n2^n} \frac{k - 1}{2^n}\I\left\{\frac{k - 1}{2^n} \leq \xi < \frac{k}{2^n}\right\} + n\I\left\{\xi \geq n\right\}.
	\]
	Непосредственная проверка убеждает в том, что \(\xi_n(\omega) \uparrow \xi(\omega)\) для любого \(\omega \in \Omega\).
\end{itemize}
