\section {Лекция от 15.11.2016}

\subsection{Свойства математического ожидания для простых случайных величин.}

\begin{enumerate}
    \item Линейность. \par
    Если $ \xi, \eta $ --- простые случайные величины, есть некоторые константы $ a, b \in \R $, то 
    \[
        \E{a\xi + b\eta} = a\E{\xi} + b\E{\eta}.
    \] 
        \begin{proof}
            Введем обозначение \(\delta = a\xi + b\eta \).
            Пусть \(x_1, \ldots, x_k \) и \(y_1, \ldots, y_m \) --- все значения случайных величин $ \xi $ и $ \eta $ соответственно, а \(z_1, \ldots, z_n \) --- все значения $ \delta $. Тогда:
            \begin{multline*}
                \E{a\xi + b\eta} = \E{\delta} = \sum\limits_{i = 1}^{n}z_i\Pr{\delta = z_i} = \sum\limits_{i = 1}^{n}z_i
                \Pr{\bigcup\limits_{\substack{t, j:\\ ax_t + by_j = z_i}}\{\xi = x_t, \eta = y_j \}} = \\ =
                \sum\limits_{i = 1}^{n}z_i\sum\limits_{\substack{t, j:\\ ax_t + by_j = z_i}}\Pr{\xi = x_t, \eta = y_j} = \\ =
                \sum\limits_{i = 1}^{n}\sum\limits_{\substack{t, j:\\ ax_t + by_j = z_i}}(ax_t + by_j)\Pr{\xi = x_t, \eta = y_j} = 
                \sum\limits_{t = 1}^{k}\sum\limits_{j = 1}^{m}(ax_t + by_j)\Pr{\xi = x_t, \eta = y_j} = \\ =
                \sum\limits_{t = 1}^{k}\sum\limits_{j = 1}^{m}ax_t\Pr{\xi = x_t, \eta = y_j} +
                \sum\limits_{t = 1}^{k}\sum\limits_{j = 1}^{m}by_j\Pr{\xi = x_t, \eta = y_j} = \\ =
                \sum\limits_{t = 1}^{k}ax_t\Pr{\xi = x_t} + \sum\limits_{j = 1}^{m}by_j\Pr{\eta = y_j} = a\E{\xi} + b\E{\eta}.
            \end{multline*}
        \end{proof}
    \item Сохранение относительного порядка: \par
     Если $ \xi, \eta $ --- простые случайные величины, тогда:
    \begin{enumerate}[label = (\alph*)]
        \item  \(\xi \geq 0  \implies \E{\xi} \geq 0 \)
        \item \(\xi \geq \eta \implies \E{\xi} \geq \E{\eta} \).
    \end{enumerate}
       \begin{proof} 
           Докажем свойства по порядку:
           \begin{enumerate}[label = (\alph*)]
               \item
               Если \(x_1, \ldots, x_k \) --- все значения $ \xi \geq 0 $, то \(\forall i = 1, \ldots, k \ x_i \geq 0 \). Значит, \(\E{\xi} = \sum\limits_{i = 1}^{k}x_i\Pr{\xi = x_i} \geq 0. \)
               \item
               Введем величину \(\delta = \xi - \eta \). Так как, \(\xi \geq \eta \), \(\delta \geq 0 \). Тогда по первому пункту \(\E{\delta} \geq 0 = \E{\xi} - \E{\eta} \).
           \end{enumerate}
       \end{proof}
    \item 
    \begin{lemma}
        Пусть \(\eta, \{\xi_n, n \in \N \} \) --- простые случайные величины, причем \(\xi_n \uparrow \xi,\ \xi_n\geq0 \ \forall n \in \N,\ \eta \leq \xi,\ \eta \geq 0 \). Тогда \[\E{\eta} \leq \lim\limits_{n}\E{\xi_n}. \] 
    \end{lemma}
        \begin{proof}
            Возьмем любое \(\alpha > 0 \) и положим \[A_n = \{\omega : \xi_n(\omega) \geq \eta(\omega) - \alpha \} .\]
            Раз \(\xi_n \uparrow \xi\), то \(A_n \uparrow \Omega \).
            По теореме о непрерывности вероятностной меры \(\Pr{A_n}~\to~1\).
            Рассмотрим математическое ожидание случайной величины \(\xi_n \):
            \[\E{\xi_n} = \E{\xi_n I_{A_n}} + \E{\xi_n I_{\overline{A_n}}} \geq \E{\xi_n I_{A_n}} \geq \E{(\eta - \alpha) I_{A_n}} = \E{\eta} - \E{\eta I_{\overline{A_n}}} - \E{\alpha I_{A_n}}. \]
            Введем обозначение: \(c = \max\limits_{\omega \in \Omega}\eta(\omega) \), тогда
            \[\E{\eta} - \E{\eta I_{\overline{A_n}}} - \E{\alpha I_{A_n}} \geq \E{\eta} - c\E{I_{\overline{A_n}}} - \alpha\E{I_{A_n}} \geq \E{\eta} - c\Pr{\overline{A_n}} - \alpha. \]
            Раз для любого $ n $, \(\xi_n \leq \xi_{n + 1}\), то \(\E{\xi_{n}} \leq \E{\xi_{n + 1}}\). Следовательно, существует \(\lim\limits_{n}\E{\xi_n}. \) 
            Перейдем к пределу в неравенстве выше: \[\lim\limits_{n}\E{\xi_n} \geq \E{\eta} - \alpha. \]
            В силу произвольности \(\alpha > 0 \), \(\lim\limits_{n}\E{\xi_n} \geq \E{\eta}. \)
        \end{proof}    
    \begin{consequence}[Корректность определения матожидания.]
        Для неотрицательных случайных величин, матожидание определено корректно.
    \end{consequence}
    \begin{proof}
        Нужно показать, что предел не зависит от выбора последовательности простых случайных величин \(0 \leq \xi_n \uparrow \xi. \) Пусть есть другая такая последовательность, \(0 \leq \eta \uparrow \xi. \) Тогда по лемме, для любого натурального $ m $ \(\lim\limits_{n}\E{\xi_n} \geq \E{\eta_m}. \) Значит, \(\lim\limits_{n}\E{\xi_n} \geq \lim\limits_{m}\E{\eta_m}. \)
        С другой стороны, для любых натуральных $ n $, \[\lim\limits_{m}\E{\eta_m} \geq \E{\xi_n} \implies \lim\limits_{m}\E{\eta_m} \geq \lim\limits_{n}\E{\xi_n}. \]
        Следовательно, \(\lim\limits_{m}\E{\eta_m} = \lim\limits_{n}\E{\xi_n}. \)
    \end{proof}
\end{enumerate}

\subsection {Свойства математического ожидания в общем случае.}
    Пусть $ \xi $ --- произвольная случайная величина. Введем обозначения: \(\xi^+ = \max(\xi, 0) \) и \(\xi^- = -\min(\xi, 0). \)
    %и \(\xi^- = \max(-\xi, 0). \)

\begin{definition}
    Говорят, что математическое ожидание \(\E{\xi} \) случайной величины $ \xi $ \emph{существует}, или \emph{определено}, если по крайней мере одна из величин $ \E{\xi^+} $ или $ \E{\xi^-} $ конечна. В этом случае по \emph{определению} полагают \( \E{\xi} = \E{\xi^+} - \E{\xi^-} \).    
    Есть четыре комбинации значений $ \E{\xi^+} $ и $ \E{\xi^-} $. Разберем их:
    \begin{enumerate}
        \item Если $ \E{\xi^+} $ и $ \E{\xi^-} $ --- конечны, то \( \E{\xi} = \E{\xi^+} - \E{\xi^-} \).
        \item Если  $ \E{\xi^+} = \infty $, $ \E{\xi^-} $ --- конечно, то $ \E{\xi} = +\infty $.
        \item Если  $ \E{\xi^+} $ --- конечно, $ \E{\xi^-} = \infty $, то $ \E{\xi} = -\infty $.
        \item Если  $ \E{\xi^+} = \infty $ и $ \E{\xi^-} = \infty $, то $ \E{\xi} $ \emph{не определено}.
    \end{enumerate}
\end{definition}

Итак, перейдем к рассмотрению свойств математического ожидания случайных величин в общем случае. Но перед этим введем еще одно важное понятие, которое будет использовать для целой группы утверждений в следующей теореме. 

\begin{definition}
    Будем говорить, что некоторое свойство $ A $ выполнено \emph{$ \Pr - $почти навеное}, если существует такое множество $ N $ с $ \Pr(N) = 0 $ такое, что это свойство $ A $ выполнено для каждой точки $ \omega = \Omega \setminus N $. Также можно определить, что событие $ A $ происходит \emph{почти наверное}, если $ \Pr{A} = 1. $ Обозначение: $ A\as.$
\end{definition}

\begin{theorem}[Свойства математического ожидания в общем случае.]
    Если \(\E{\xi} \) --- математическое ожидание случайной величины \(\xi \), то выполняются следущие свойства:
    \begin{enumerate}
        \item Если \(c \in \R,\ \E{\xi} \) --- конечно, тогда \[\E{c\xi} = c\E{\xi}. \]
        \item Пусть \(\xi \leq \eta \) и \(\E{\xi}, \E{\eta} \) --- конечны. Тогда \[\E{\xi} \leq \E{\eta}. \]
        \item Если \(\E{\xi}\) и \(\E{\eta}\) --- конечны, то \(\E{\xi + \eta} \) --- конечно и \[\E{\xi + \eta} = \E{\eta} + \E{\xi}. \]
        \item Если \(\E{\xi} \) --- существует, тогда \[\left|\E{\xi}\right| \leq \E{|\xi|}. \]
        \item 
            \begin{enumerate}[label = (\alph*)]
                \item Пусть \(0 \leq \xi \leq \eta \) и \(\E{\eta} \) --- конечно, тогда \(\E{\xi} \) также конечно.
                \item Пусть \(|\xi| \leq \eta \) и \(\E{\eta} \) -- конечно, тогда \(\E{\xi} \) также конечно.
            \end{enumerate}
        \item Если \(\xi = 0\as \), то \(\E{\xi} = 0. \)
        \item Если \(\xi = \eta\as \), и \(\E{\eta} \) --- конечно, то \(\E{\xi} = \E{\eta}. \)
        \item Если \(\xi \geq 0 \) и \(\E{\xi} = 0 \), то \(\xi = 0\as. \)
        \item Пусть \(\forall A \in \F,\ \E{\xi I_A} \leq \E{\eta I_A} \) и \(\E{\xi},\ \E{\eta} \) --- конечны. Тогда \(\xi \leq \eta\as. \) 
    \end{enumerate}
\end{theorem}
    \begin{proof}
        Докажем первую группу свойств:
        \begin{enumerate}
            \item Для простых случайных величин было доказано в первой части лекции. В общем же случае надо рассмотреть представление \(\xi = \xi^+ - \xi^- \) и заметить, что для \(c \geq 0, \ (c\xi)^+ = c\xi^+, \ (c\xi)^- = c\xi^- \), а для \(c < 0,\  (c\xi)^+ = -c\xi^-,\ (c\xi)^- = -c\xi^+ .\)
            \item Для простых случайных величин было доказано в первой части лекции. 
            Рассмотрим общий случай. Пусть теперь $ \E{\xi} > -\infty $, тогда \(\E{\xi^-} < \infty \). Если \(\xi \leq \eta \), то \(\xi^+ \leq \eta^+,\ \xi^- \geq \eta^- \). Тогда \(\E{\eta^-} \leq \E{\xi^-} < \infty \), следовательно, \(\E{\eta} \) определено и \(\E{\xi} = \E{\xi^+} - \E{\xi^-} \leq \E{\eta^+} - \E{\eta^-} = \E{\eta}. \) Аналогичным образом рассматривается случай, когда \(\E{\eta} < \infty. \)
            \item Случай с простыми случайными величинами разбирался в первой части лекции. В общем же случае, когда \(\E{|\xi|} < \infty, \ \E{|\eta|} < \infty \), сводится к рассмотренному, если воспользоваться тем, что \(\xi = \xi^+ - \xi^-,\ \eta = \eta^+ - \eta^-,\ \xi^+ \leq |\xi|,\ \xi^- \leq |\xi|,\ \eta^+ \leq |\eta|,\ \eta^- \leq |\eta|. \) 
            \item Поскольку \(-|\xi| \leq \xi \leq |\xi| \), по первым двум свойствам \(-\E{\left|\xi\right|} \leq \E{\xi} \leq \E{\left|\xi\right|}. \)
        \end{enumerate}
       Теперь докажем вторую группу свойств, связанных с понятием ``\emph{$ \Pr - $почти навеное}'':
       \begin{enumerate}
           \setcounter{enumi}{5}
           \item В самом деле, если $ \xi $ -- простая случайная величина, тогда она по определению разбивается в следующую сумму: \(\xi = \sum x_k I_{A_k}(\omega) \) и $ x_k \neq 0 $. По условию, \(\xi = 0\as \), следовательно для любых событий \(A_k = \{\xi = x_k\} \), \(\Pr{A_k} = 0 \), а значит, \(\E{\xi} = 0\). Если же \(\xi \geq 0 \) --- неотрицательная случайная величина, то для любой последовательности $ \{\xi_n \},\ 0 \leq \xi_n \uparrow \xi $, \[\Pr{\xi_n} \geq \Pr{\xi = 0} = 1 \implies \xi_n = 0\as. \] Значит, \[\E{\xi_n} = 0,\  \forall n \implies \E{\xi} = \lim\limits_{n} \E{\xi_n} = 0.\]
           Если \(\xi = 0\as \), то \[\xi^+ = 0\as,\ \xi^- = 0\as, \implies \E{\xi^+} = 0\as,\ \E{\xi^-} = 0\as \implies \E{\xi} = \E{\xi^+} - \E{\xi^-} = 0.  \]
           \item В самом деле, пусть $ N = \{\omega: \xi \neq \eta \}. $ Тогда $ \Pr{N} = 0 $ и \(\xi = \xi I_N + \xi I_{\overline{N}},\ \eta = \eta I_N + \eta I_{\overline{N}} = \eta I_N + \xi I_{\overline{N}}. \) По свойствам (3) и (6) \(\E{\xi} = \E{\xi I_N} + \E{\xi I_{\overline{N}}} = \E{\xi I_N} = \E{\eta I_{\overline{N}} }. \) Но \(\E{\eta I_N} = 0 \), поэтому по свойству (3) \(\E{\xi} = \E{\eta I_{\overline{N}}} + \E{\eta I_N} = \E{\eta}. \)
           \item Пусть $ \xi $ --- простая случайная величина со значениями \(x_1, \ldots, x_k. \) Тогда \(x_i \geq 0 \) и \(\E{\xi} = \sum\limits_{i = 1}^{k} x_i \Pr{\xi = x_i} = 0. \) Следовательно, для любого $ i $, такого что $ x_i \neq 0 $ выполнено \(\Pr{\xi = x_i} = 0 \implies \Pr{\xi = 0} = 1. \)
           Если \(\xi \geq 0 \), то \(\forall n \in \N \) рассмотрим \(A_n = \{\xi \geq \frac{1}{n} \} \). \[\Pr(A_n) = \E{I_{A_n}} \leq \E{n\xi I_{A_n}} \leq \E{n\xi} \leq n\E{\xi} = 0.\] Но \(A_n \downarrow \{\xi > 0 \} \implies \Pr{\xi > 0} = \lim\limits_{n}\Pr{A_n} = 0. \)
           \item  В самом деле, пусть \(B = \{\omega : \xi(\omega) > \eta(\omega) \}. \) Тогда \(\E{\eta I_B} \leq \E{\xi I_M} \leq \E{\eta I_B}  \) и, значит, \(\E{\xi I_B} = \E{\eta I_b} \). В силу свойства аддитивности (3) \(\E{(\xi - \eta) I_B} = 0 \) и по свойству (8) \((\xi - \eta)I_B = 0\as \), откуда \(\Pr{B} = 0. \)
       \end{enumerate}
    \end{proof}