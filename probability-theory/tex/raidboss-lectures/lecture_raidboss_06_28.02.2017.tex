\section{Лекция от 28.02.2017}

\renewcommand{\labelitemii}{$\bullet$}

\subsection{Общее число частиц после $n$-го хода}

Будем считать, что $Y_n \to Y$ ($Y$ может быть бесконечностью, сходимость
имеется в виду предельная, то есть смотрят на предел $Y_n$ и вероятность каждого значения 
(в том числе и бесконечность) --- это вероятность принять это значение $Y_n$-ыми).
Давайте
поймём, что $\rho(z)$ --- производящая функция величины $Y\I\{Y < +\infty\}$. 
Действительно, если ограничить $Y$ конечными значениями, то получим, что

\[
  \phi_{Y\I\{Y < +\infty\}}(z) = \sum\limits_{k = 0}^{+\infty} z^k \Pr{Y = k}
\]

А теперь давайте поймём, что это в точности $\rho(z)$ при $z \in [0, 1]$.
В единице как раз вероятность вырождения, тут всё отлично. В остальных точках
это предел производящих функций, то есть предел $\E{z^{Y_n}} \to \E{z^{Y \I\{Y < +\infty\}}}$,
действительно, у величин $Y_n$ есть сходимость по распределению
к $Y \I\{Y < +\infty\}$, а $z^k$ при $z \in [0, 1)$ ограниченная функция. Поэтому
в пределе как раз и получим $\rho(z)$.

Давайте рассмотрим пример после не самых тривиальных следствий.

\begin{example}
  $\xi \sim \mathrm{Geom}(p), p \in (0, 1)$, причём сделаем его с нуля ($\Pr{\xi = k} =
  p(1 - p)^k, k \in \Z_+$). Пусть $\{X_n, n \in \Z_+\}$ --- ветвящийся процесс
  с законом размножения $\xi$. Интересно было бы найти $\rho, \Pr{Y = k}$. Давайте
  этим и займёмся.

  \[
    \phi_{\xi}(z) = \E{z^\xi} = \sum\limits_{k = 0}^{+\infty} z^k(1-p)^kp = 
    \frac{p}{1 - z(1 - p)}
  \]

  Откуда из леммы о пределе характеристических функций получаем:

  \[
    \rho = \frac{zp}{1 - \rho(1 - p)}
  \]

  После недолгих вычислений с учётом, что $p \neq 1$ получим:

  \[
    \rho(z) = \frac{1 \pm \sqrt{1 - 4zp(1 - p)}}{2(1 - p)}
  \]

  Надо теперь выбрать, какой знак нам подойдет. Если подставить $z = 0$, то
  по лемме предел должен быть равен нулю ($Y_n > 0$ всегда из-за начальной частицы).

  Поэтому надо выбрать знак <<минус>>, чтобы сошлось. Отлично, мы знаем производящую
  функцию, давайте теперь вычислять коэффициенты через ряд Тейлора и производные.

  \begin{multline}
    \rho(z) = -\sum\limits_{k = 1}^{+\infty} \binom{1/2}{k}(-4 p(1 - p))^k \frac{1}{2(1 - p)}z^k =
    -\sum\limits_{k = 1}^{+\infty}\frac{\frac12\left(\frac{1}{2} - 
    1\right)\cdot\ldots\cdot(\frac{1}{2} - k + 1)}{2k!(1 - p)}2^k (-4p(1 - p))^k z^k =\\=
    \sum\limits_{k = 1}^{+\infty}\frac{(2k - 3)!!2^k}{k!}(p(1 - p))^k \frac{1}{2(1 - p)} z^k =
    \sum\limits_{k = 1}^{+\infty}\frac12 \binom{2k}{k} \frac{1}{2k - 1} p^k(1 - p)^{k - 1} z^k
  \end{multline}

  Откуда $\Pr{Y = k} =\frac12 \binom{2k}{k} \frac{1}{2k - 1} p^k(1 - p)^{k - 1}$.

\end{example}

На самый конец давайте ещё вычислим математическое ожидание частиц при условии, 
что процесс конечен. Это даёт хорошее приближение $Y_n$, но так изворачиваемся
мы потому, что через производящие функции легче посчитать это значение у $Y$,
нежели возиться с $Y_n$ и пределами

\[
  \E{Y \given Y < +\infty} = \sum\limits_{k = 0}^{+\infty} k \Pr{Y = k \given Y < +\infty} =
  \sum\limits_{k = 0}^{+\infty} \frac{k \Pr{Y = k}}{q} = 
  \frac{\rho'(1)}{q}
\]

Теперь вычислим производную слева и справа у давно нам известного равенства
$\rho(z) = z\phi_{\xi}(\rho(z))$ и сразу подставим единицу:

\[
  \rho'(1) = \phi_{\xi}(\rho(1)) + \rho'(1)\phi_{\xi}'(\rho(1))
  \implies
  \rho'(1) = \frac{q}{1 - \phi_{\xi}'(\rho(1))}
\]

Откуда наконец-то вычисляется наш финальный результат:

  
\[
  \E{Y \given Y < +\infty} = \frac{1}{1 - \phi_{\xi}'(q)}
\]

На этом случайные процессы Гальтона-Ватсона заканчиваются.

\subsection{Случайные графы}

Случайные графы можно описать различными способами. Существует несколько
моделей создавать графы, о которых мы поговорим немного позже. Случайные графы 
нашли практическое применение во всех областях, где нужно смоделировать 
сложные сети --- известно большое число случайных моделей графов, отражающих 
разнообразные типы сложных сетей в различных областях. Но, к сожалению, граф,
задающий интернет, не описывается какой-то простой моделью. Об этом мы тоже
поговорим в этой лекции.

\subsection{Классические модели}

Существуют две классические модели, задающие случайный граф. Вообще, мы всегда
будем работать с графом на $n$ вершинах и каким-то образом выбирать остовный
подграф на этих вершинах. Напомним, что остовный подграф --- подграф на этих
же вершинах, но с каким-то (возможно пустым) подмножеством ребер.

\begin{itemize}
  \item[1.] \textbf{Биномиальная}
  \begin{definition}
    $G(n, p)$ --- случайный граф, который имеет следующее распределение для любого остовного
    подграфа $G$:
    \[
      \Pr{G(n, p) = G} = p^{|E(G)|}(1 - p)^{\binom{n}{2} - |E(G)|}
    \]
  \end{definition}

  Другими словами, любое ребро включается в случайный граф с вероятностью $p$.

  \item[2.] \textbf{Равномерная}
  \begin{definition}
    Пусть $\mathfrak{G}_m$ --- все остовные подграфы с количеством ребер $m$.
    Тогда $G(n, m)$ --- случайный граф, который имеет равномерное распределение для любого
    $G \in \mathfrak{G}_m$
    \[
      \Pr{G(n, m) = G} = \frac{1}{\binom{\binom{n}{2}}{m}}
    \]
  \end{definition}
\end{itemize}

Несложно заметить, что если $p(n)$ и $m(n)$ таковы, что $\binom{n}{2}p \sim m$,
то вероятности будут не сильно отличаться, т.е. $\Pr{G(n, p) \text{ обладает свойством $Q$}} \sim
\Pr{G(n, m) \text{ обладает свойством $Q$}}$, потому что мы выбираем в среднем
около $m$ ребер. Конечно, это нестрогое рассуждение, но эквивалентность этих
моделей и вправду (как мы увидим позднее) почти всегда выполняется для всех
адекватных свойств $Q$.

Приведем ещё некоторые (менее популярные) модели случайных графов.

\begin{itemize}
  \item[3.] \textbf{Случайный $d$-регулярный граф}

  Просто равномерно строится случайный $d$-регулярный граф. Один из хороших
  способов строить такой граф --- взять $nd$ точек, после этого разбить по $n$ групп
  по $d$ точек в каждую группу. После этого надо взять произвольное совершенное
  паросочетание так, чтобы в каждой группе не было ребер между собой. После этого
  надо сжать каждую группу в одну вершину и сказать, что мы построили граф.
  Оставим упражнение читателю для доказательства, что так равновероятно будет
  получен любой регулярный граф.

  \item[4.] \textbf{Случайный двудольный граф}

  Тут всё просто, если есть полный граф $K_{n, m}$, то с вероятностью $p$ мы
  берем каждое ребро.
\end{itemize}

\subsection{Случайный процесс на графах}

Генерировать графы можно и как случайный процесс. То есть добавлять рёбра 
(или даже несколько ребер) в какие-то
моменты времени. Существуют два вида времени, для которых хоть сколько-нибудь
осмысленно рассматривать случайный процесс на графах.

\begin{itemize}
  \item[1.] \textbf{Дискретное время}
  Тут достаточно всё интуитивно. Каждый момент времени мы добавляем какое-то
  ребро, которое ещё не было добавлено до этого.
  \begin{definition}
    $\tilde{G} = (\tilde{G}(n, m), m = 0, \ldots,
    m = \binom{n}{2})$ таков, что $\tilde{G}(n, 0)$ пуст и 
    $\tilde{G}(n, m) = \tilde{G}(n, m - 1)$ плюс какое-то ребро из тех, кто не вошёл
    в $\tilde{G}(n, m)$.
  \end{definition}

  Практически очевидно, что $\tilde{G}(n, m) \eqdist G(n, m)$ (можно воспользоваться
  идеей перенумерации ребер).

  Но такой процесс позволяет нам позволяет смотреть динамику получившихся графов.
  Давайте анонсируем некоторые результаты, один из которых мы позже докажем.

  Пусть $\tau_1$ --- первый момент, когда нет изолированных вершин, $\tau_2$ ---
  первый момент, когда граф становится связным, $\tau_3$ --- когда степень каждой
  вершины
  становится хотя бы двойка, а $\tau_4$ --- первый момент, когда граф становится
  гамильтоновым.

  Ясно, что между этими величинами можно поставить следующие соотношения:
  \begin{itemize}
    \item $\tau_1 \leq \tau_2$, т.к в связных графах нет изолированных вершин;
    \item $\tau_1 \leq \tau_3$, т.к если у каждой вершины хотя бы степень 2, то нет ни одной изолированной;
    \item $\tau_3 \leq \tau_4$, т.к гамильтонов цикл уж точно нам даёт, что из 
    каждой вершины выходит хотя бы 2 ребра.
  \end{itemize}
  Но связь между этими случайными величинами более тонкая.
  \begin{theorem}
    \begin{align}
      \Pr{\tau_1 = \tau_2} = 1, n \to +\infty\\
      \Pr{\tau_3 = \tau_4} = 1, n \to +\infty
    \end{align}
  \end{theorem}

  Пока оставим без доказательства, но первый пункт мы точно докажем в курсе, второй
  доказывается тяжело.

  Если так вдуматься, то получается, что как только перестают существовать изолированные
  вершины, так сразу граф становится связен. И как только степень вершин хотя бы
  два, то граф уже гамильтонов. Возможно это противоречит интуиции, но так и есть и 
  вскоре мы с этим разберемся.

  \item[2.] \textbf{Непрерывное время}

  \begin{definition}
    Занумеруем как-то ребра полного графа $K_n$, тогда $\tilde{G} = (G(n, t), t \geq 0)$
    называется случайным процессом на графе с непрерывным временем и $\tilde{G}(n, t)$
    состоит из тех ребер $i$ для которых $t(i) \leq t$.
  \end{definition}

  $t(i)$ можно выбирать как угодно, возможно из какого-то распределения случайно
  или только некоторые из них случайно. Это понятие достаточно хорошо коррелирует
  с биномиальной моделью, а именно $\tilde{G}(n, t) \eqdist G(n, p)$ при $p = 
  \Pr{t(i) \leq t}$ --- если каждое $t(i)$ одинаково выбирается случайно каким-либо
  образом.

  \item[3.] \textbf{Triangle free process}

  То же, что и в дискретном процессе, только мы не берем на шаге ребро, если
  оно образовывает треугольник. И останавливаемся, когда такой ход сделать нельзя.

  Данный процесс получил широкое распространение из-за теории Рамсея, так как
  размер независимого множества получается не самым маленьким (порядка $\Omega(\sqrt{n \ln n})$) и
  одновременно получилась точная оценка ($\mathcal{O}(\sqrt{n \ln n})$), что дало огромное
  количество значений для чисел Рамсея $R(3, n)$ вплоть до $n = 9$.
\end{itemize}

\subsection{Граф интернета}

Одним из самых важных графов для поисковых компаний на данное время --- является
граф интернета. Но, к сожалению, его не удалось как-то просто моделировать по многим
причинам. Опишем какие свойства у графа интернета есть (мы рассматриваем модель,
что вершины это ссылки, а нахождение каких-то ссылок на другие сайты --- это рёбра).

\begin{itemize}
  \item Эмпирически было выяснено, что ребер в таком графе равно $\mathcal{O}(n)$.
  То есть граф является разреженным;
  \item Степенной закон. Вершин степени $d$ в таком равно примерно $\frac{cn}{d^{2.3}}$.
  Гораздо удивительнее здесь константа 2.3, которая не меняется с течением времени.
  \item Константный малый диаметр. Как и в мире существует гипотеза о шести рукопожатиях,
  так и в веб-графе существует закон шести кликов --- это факт, состоящий в том,
  что диаметр веб-графа равен шести.
  \item В 1999 году двое исследователей --- А. Л. Барабаши и Р. Альберт --- 
  предложили крайне простую идею, которая, однако ж, оказалась весьма продуктивной.
  Идея заключалась в том, что, когда новый сайт появляется на свет, он, скорее 
  всего, <<предпочитает>> сослаться на те сайты, которые и без того уже многими 
  цитированы. То есть с каждым новым сайтом добавляются ссылки, пропорциональному
  степени входящих и исходящих ребер того или иного сайта.
  \item Будем называть два узла соседями, если существует связь между ними. 
  Для веб-графа характерно, что два узла, соседних к какому-либо узлу, 
  часто также являются соседями между собой. Чтобы охарактеризовать это явление 
  и был предложен кластерный коэффициент $C_i$ узла $i$.
  Предположим, что узел имеет степень $k_i$ , это значит, что у него $k_i$ соседей
  и между ними может быть максимум $k_i (k_i - 1)/2$ связей. Тогда

  $$ C_i = \frac{2 n_i}{k_i (k_i - 1)},$$
  
  где $n_i$ --- число связей между соседями узла $i$.

  У веб-графа, тем не менее, наблюдается постоянный и немаленький
  кластерный коэффициент.
\end{itemize}

Сложность заключается в том, что простыми моделями такой граф не описать, так
как у каждой из них существуют те или иные отклонения от пунктов, которые написаны
выше. Кроме того, ясно, что в гипотетических моделях веб-графа
у более старых вершин гораздо больше шансов
иметь большую входящую степень, нежели у более новых вершин. Это заведомо плохо
согласуется с новостными <<взрывами>>, которые ежедневно случаются в интернете: 
едва появляется страница с важной новостью, как на нее приходят тысячи ссылок.
Мы уж не говорим о том, что сайты, страницы и ссылки на них зачастую умирают,
и это тоже сложность для отражения в моделях.

\subsection{Монотонные свойства}

Но мы всё же остановим свой взгляд на классических моделях графов. Чтобы
что-то доказывать, нужно как-то классифицировать свойства графов, так как у нас
уже есть различные процессы, которые добавляют рёбра или, наоборот, убирают их.

Напомним, что свойством мы называем любое подмножество графов на $n$ вершинах.

\begin{definition}
  Свойство $Q$ называется \textit{возрастающим}, если для любого $G_1$ обладающим
  этим свойством и тем, что $G_1 \subseteq G_2$ следует, что $G_2$ тоже обладает
  этим свойством.
\end{definition}

\begin{example}
  Примеров достаточно много, таких как:
  \begin{itemize}
    \item Связность;
    \item Отсутствие изолированных вершин;
    \item Наличие любой клики.
  \end{itemize}
\end{example}

\begin{definition}
  Свойство $Q$ называется \textit{убывающим}, если для любого $G_1$ обладающим
  этим свойством и тем, что $G_1 \supseteq G_2$ следует, что $G_2$ тоже обладает
  этим свойством.
\end{definition}

\begin{example}
  Здесь немного другого типа примеры, которые выполняются для пустых графов.
  \begin{itemize}
    \item Несвязность;
    \item Планарность;
    \item Ацикличность;
    \item Двудольность.
  \end{itemize}
\end{example}

\begin{definition}
  Свойство $Q$ называется \textit{выпуклым}, если для любого $G_1$ и $G_2$ обладающими
  этим свойством и $G_3$ таким, что $G_1 \subseteq G_3 \subseteq G_2$ следует,
  что $G_3$ обладает этим свойством.
\end{definition}

\begin{example}
  Число изолированных вершин равно $k$ (оно не монотонное).
\end{example}

Для разнообразия приведем пример не выпуклого и не монотонного свойства --- 
<<максимальная компонента связности является деревом>>. Желающие могут самостоятельно
придумать примеры, которые показывают, что это свойство не монотонное и не выпуклое.

