\section{Лекция от 27.01.2017}
\subsection{Сходимость случайных векторов}
Ранее мы изучали сходимость для случайных величин. Обобщим это для случайных векторов, введя те же типы сходимости.
\begin{definition}
	Пусть \(\{\xi_n \mid n \in \N\}, \xi\)~--- случайные векторы из \(\R^m\). Тогда
	\begin{enumerate}
		\item Будем говорить, что последовательность \(\{\xi_n\}\) сходится к \(\xi\) \emph{почти наверное} (или с вероятностью 1), если
		\[
		\Pr{\lim\limits_{n \to \infty}\xi_n = \xi} = 1.
		\]
		\item Будем говорить, что последовательность \(\{\xi_n\}\) сходится к \(\xi\) \emph{по вероятности}, если для любого \(\epsilon > 0\)
		\[
		\lim\limits_{n \to \infty} \Pr{\left\|\xi_n - \xi\right\|_2 \geq \epsilon} = 0
		\]
		В данном случае \(\left\|x\right\|_2 = \sqrt{x_1^2 + \dots + x_m^2}\).
		\item Будем говорить, что последовательность \(\{\xi_n\}\) сходится к \(\xi\) \emph{по распределению}, если для любой непрерывной ограниченной функции \(h : \R^m \mapsto \R\) выполнено, что
		\[
		\lim\limits_{n \to \infty} \E{h(\xi_n)} = \E{h(\xi)}.
		\]
	\end{enumerate}
\end{definition}

Так, стоп. Первые два похожи на определения в одномерном случае, но третье кардинально отличается от одномерного случая~--- ведь там была сходимость функций распределения. Сразу возникает вопрос: а совпадают ли эти определения для одномерного случая? Ведь случайная величина тоже является случайным вектором. Оказывается, что всё хорошо. Это утверждает следующая
\begin{theorem}[об эквивалентности определений сходимости по распределению]
	Пусть \(\{\xi_n \mid n \in \N\}, \xi\)~--- случайные векторы из \(\R^m\).
	\begin{enumerate}
		\item Пусть \(m = 1\). Тогда \(\xi_n \xrightarrow{d} \xi\) (в смысле случайных величин), тогда и только тогда, когда для любой непрерывной ограниченной функции \(h : \R^m \mapsto \R\) выполнено, что \(\lim\limits_{n \to \infty} \E{h(\xi_n)} = \E{h(\xi)}\).
		\item Если \(m > 1\) и \(F_{\xi}\) непрерывна в \(\R^m\), то \(\xi_n \xrightarrow{d} \xi\) равносильно тому, что для любого \(x \in \R^m\) \(\lim\limits_{n \to \infty} F_{\xi_n}(x) = F_{\xi}(x)\).
	\end{enumerate}
\end{theorem}
\begin{proof}
	Как обычно, принимаем без доказательства.
\end{proof}
Следующиее упражнение может быть весьма полезно для доказательства некоторых фактов.
\begin{exercise}
	Пусть \(\{\xi_n \mid n \in \N\}, \xi\)~--- случайные векторы в \(\R^m\) и \(\xi_n = (\xi_n^{(1)}, \xi_n^{(2)}, \dots, \xi_n^{(m)})\). Тогда
	\begin{enumerate}
		\item \(\xi_n \asto \xi \iff \forall i \in \{1, 2, \dots, m\}\ \xi_n^{(i)} \asto \xi^{(i)}\)
		\item \(\xi_n \prto \xi \iff \forall i \in \{1, 2, \dots, m\}\ \xi_n^{(i)} \prto \xi^{(i)}\)
		\item \(\xi_n \dto \xi \implies \forall i \in \{1, 2, \dots, m\}\ \xi_n^{(i)} \dto \xi^{(i)}\), причём в обратную сторону это неверно.
	\end{enumerate}
\end{exercise}

Вспомним, что в одномерном случае существовала <<иерархия>> сходимостей: почти наверное \(\Rightarrow\) по вероятности \(\Rightarrow\) по распределению. Оказывается, что она остаётся верна и в многомерном случае.
\begin{lemma}[о взаимосвязи видов сходимости]
	Пусть \(\{\xi_n \mid n \in \N\}\), \(\xi\)~--- случайные векторы в \(\R^m\). Тогда
	\begin{enumerate}
		\item \(\xi_n \asto \xi \implies \xi_n \prto \xi\)
		\item \(\xi_n \prto \xi \implies \xi_n \dto \xi\)
	\end{enumerate}
\end{lemma}
\begin{proof}
	Вообще говоря, тут и доказывать нечего. Первый пункт автоматически следует из утверждения о равносильности общей и покоординатной сходимости для сходимостей почти наверное и по вероятности. Доказательство второго пункта аналогично одномерному случаю.
\end{proof}

\subsection{Теорема о наследовании сходимости, лемма Слуцкого и их применения}
Сформулируем и докажем две важные теоремы.
\begin{theorem}[о наследовании сходимости]
	Пусть \(\{\xi_n \mid n \in \N\}\), \(\xi\)~--- случайные векторы в \(\R^m\), а \(h : \R^m \mapsto \R^n\)~--- это некоторая функция.
	\begin{enumerate}
		\item Если \(\xi_n \asto \xi\) и \(h\) непрерывна почти всюду относительно распределения \(\xi\) (то есть существует \(B \in \B(\R^m)\) такое, что \(h\) непрерывна на \(B\) и \(\Pr{\xi \in B} = 1\)), то \(h(\xi_n) \asto h(\xi)\).
		
		\item Аналогично, если \(\xi_n \prto \xi\) и \(h\) непрерывна почти всюду относительно распределения \(\xi\), то \(h(\xi_n) \prto h(\xi)\).
		
		\item Если \(\xi_n \dto \xi\) и \(h\) непрерывна, то \(h(\xi_n) \dto h(\xi)\).
	\end{enumerate}
\end{theorem}
\begin{proof}
	Вообще говоря, первый и третий пункт по сложности сравнимы с проверкой определений. Самое сложное~--- это второй пункт.
	\begin{enumerate}
		\item Для начала докажем утверждение для сходимости почти наверное. Рассмотрим вероятность
		\[
			\Pr{\lim\limits_{n \to \infty} h(\xi_n) = h(\xi)}
		\]
		Заметим, что для любого \(\xi \in B\) это равносильно тому, что \(\xi_n \asto \xi\) (по непрерывности \(h\)). Тогда эту вероятность можно ограничить снизу:
		\begin{multline*}
			\Pr{\lim\limits_{n \to \infty} h(\xi_n) = h(\xi)} \geq \Pr{\lim\limits_{n \to \infty} \xi_n = \xi, \xi \in B} \\
			\geq \Pr{\lim\limits_{n \to \infty} \xi_n = \xi}\Pr{\xi \in B} = 1.
		\end{multline*}
		Это и означает, что \(h(\xi_n) \asto h(\xi)\).
		
		\item Теперь докажем утверждение для сходимости по вероятности от противного. Допустим, что \(h(\xi_n) \not\prto h(\xi)\). Тогда существуют положительные \(\epsilon_0\) и \(\delta_0\) и подпоследовательность \(\{\xi_{n_k} \mid k \in \N\}\) такие, что
		\[
			\Pr{\left\|h(\xi_{n_k}) - h(\xi))\right\| \geq \epsilon_0} \geq \delta_0
		\]
		Однако \(\xi_{n_k} \prto \xi\). Тогда есть ещё одна подпоследовательность \(\{\xi_{n_{k_s}} \mid s \in \N\}\) такая, что \(\xi_{n_{k_s}} \asto \xi\). Тогда согласно первому пункту \(h(\xi_{n_{k_s}}) \asto h(\xi)\). Но по лемме о взаимосвязи сходимостей \(h(\xi_{n_{k_s}}) \prto h(\xi)\). Получаем противоречие. Следовательно, \(h(\xi_{n}) \prto h(\xi)\).
		
		\item Предположим, что \(\xi_{n} \dto \xi\) и \(h: \R^n \mapsto \R^k\)~--- непрерывная функция. Возьмём произвольную ограниченную и непрерывную функцию \(f: \R^k \mapsto \R\). Тогда композиция \(f \circ h : \R^n \mapsto \R\)~--- ограниченная и непрерывная функция. Следовательно, так как \(\xi_{n} \dto \xi\), то
		\[
		\E{f(h(\xi_n))} \to \E{f(h(\xi))}.
		\]
		Но это означает, что \(h(\xi_{n}) \dto h(\xi)\).
	\end{enumerate}
\end{proof}

\begin{exercise}
	Предположим, что некоторый случайный вектор \(\xi\) сходится к константе по распределению. Докажите, что он сходится к ней же по вероятности.
\end{exercise}

\begin{lemma}[Слуцкий]
	Пусть \(\{\xi_{n} \mid n \in \N\}\) и \(\{\eta_{n} \mid n \in \N\}\)~--- две последовательности случайных величин. Далее, пусть существуют константа \(С\) и случайная величина \(\xi\) такие, что \(\xi_{n} \dto \xi\) и \(\eta_{n} \dto C\). Тогда
	\begin{enumerate}
		\item \(\xi_{n} + \eta_{n} \dto \xi + C\),
		\item \(\xi_{n}\eta_{n} \dto C\xi\)
	\end{enumerate}
\end{lemma}
\begin{proof}
	Докажем это утверждение только для суммы, ибо для произведения всё аналогично.
	
	Пусть \(t\)~--- это точка непрерывности \(F_{\xi + C}\). Выберем сколь угодно малый \(\epsilon\) такой, что \(t \pm \epsilon\) тоже являются точками непрерывности \(F_{\xi + C}\). Теперь рассмотрим \(F_{\xi_{n} + \eta_{n}}(t)  = \Pr{\xi_{n} + \eta_{n} \leq t}\). По формуле полной вероятности:
	\[
		F_{\xi_{n} + \eta_{n}}(t) = \Pr{\xi_{n} + \eta_{n} \leq t, \eta_{n} > t - \epsilon} + \Pr{\xi_{n} + \eta_{n} \leq t, \eta_{n} \leq t - \epsilon}.
	\]
	Увеличим вторую вероятность следующим образом: забудем про первое условие, а второе заменим на \(|\eta_{n} - t| \geq \epsilon\). Тогда
	\[
		F_{\xi_{n} + \eta_{n}}(t) \leq \Pr{\xi_{n} + \eta_{n} \leq t, \eta_{n} > t - \epsilon} + \Pr{|\eta_{n} - t| \geq \epsilon}.
	\]
	Теперь увеличим первую вероятность. Из условий следует, что \(\xi_{n} + C \leq t + \varepsilon\). Тогда заменим оба условия на это. Следовательно,
	\[
		F_{\xi_{n} + \eta_{n}}(t) \leq \Pr{\xi_{n} + C \leq t + \epsilon} + \Pr{|\eta_{n} - t| \geq \epsilon}.
	\]
	Осталось понять, куда стремятся полученные вероятности. Так как \(\xi_{n} \dto \xi\), то \(\xi_{n} + C \dto \xi + C\). Учитывая то, что \(t + \epsilon\) есть точка непрерывности, то первая вероятность стремится к \(F_{\xi + C}(t + \epsilon)\). Вторяя же вероятность стремится к нулю, так как \(\eta_{n} \dto C\) (и, следовательно, \(\eta_{n} \prto C\)).
	
	Отсюда получаем, что
	\[
		\varlimsup_{n \to \infty} F_{\xi_{n} + \eta_{n}}(t) \leq F_{\xi + C}(t + \epsilon).
	\]
	
	Аналогичными рассуждениями получаем, что
	\[
		\Pr{\xi_{n} + \eta_{n} \geq t} \geq \Pr{\xi_{n} + C \geq t + \epsilon} + \Pr{|\eta_{n} - t| \geq \epsilon}.
	\]
	Тогда
	\[
		\varlimsup_{n \to \infty} (1 - F_{\xi_{n} + \eta_{n}}(t)) \geq 1 - F_{\xi + C}(t + \epsilon) \text{ и } \varliminf_{n \to \infty} F_{\xi_{n} + \eta_{n}}(t) \geq F_{\xi + C}(t - \epsilon).
	\]
	
	Получаем следующую цепочку ограничений:
	\[
		F_{\xi + C}(t - \epsilon) \leq \varliminf_{n \to \infty} F_{\xi_{n} + \eta_{n}}(t) \leq \varlimsup_{n \to \infty} F_{\xi_{n} + \eta_{n}}(t) \leq F_{\xi + C}(t + \epsilon).
	\]
	Устремляя \(\epsilon\) к нулю и пользуясь тем, что \(t\)~--- это точка непрерывности \(F_{\xi + C}\), получаем, что
	\[
		\lim\limits_{n \to \infty} F_{\xi_{n} + \eta_{n}}(t) = F_{\xi + C}(t).\qedhere
	\]
\end{proof}
Теперь посмотрим, как применять их на практике.
\begin{problem}
	Пусть \(\{\xi_{n} \mid n \in \N\}\)~--- последовательность независимых и одинаково распределённых случайных величин, про которые известно, что \(a = \E\xi_1 < \infty\) и \(0 < \sigma^2 = \D\xi_1 < +\infty\). Введём обозначение \(S_n = \xi_{1} + \dots + \xi_{n}\). Найдите предел по распределению
	\[
		\sqrt{n}\left(\frac{n}{S_n} - \frac{1}{a}\right)
	\]
\end{problem}
\begin{proof}[Решение]
	Преобразуем это выражение следующим образом:
	\[
		\sqrt{n}\left(\frac{n}{S_n} - \frac{1}{a}\right) = -\frac{\sigma}{a}\cdot\frac{n}{S_n}\cdot\sqrt{n}\left(\frac{S_n}{n} - a\right)\frac{1}{\sigma}
	\]
	Осталось заметить следующее:
	\begin{gather*}
		\text{По теореме о наследовании сходимости и уЗБЧ }\frac{S_n}{n} \asto
		a \implies \frac{n}{S_n} \dto \frac{1}{a} \\
		\text{По ЦПТ } \sqrt{n}\left(\frac{S_n}{n} - a\right)\frac{1}{\sigma} \dto \mathcal{N}(0, 1)
	\end{gather*}
	Следовательно, по лемме Слуцкого
	\[
		\sqrt{n}\left(\frac{n}{S_n} - \frac{1}{a}\right) \dto -\frac{\sigma}{a^2}\mathcal{N}(0, 1) = \mathcal{N}\left(0, \frac{\sigma^2}{a^4}\right).\qedhere
	\]
\end{proof}

Теперь обобщим закон больших чисел:
\begin{theorem}[усиленный ЗБЧ для векторов]
	Пусть \(\{\xi_{n} \mid n \in \N\}\)~--- последовательность независимых и одинаково распределённых случайных \(m\)-мерных векторов. Далее, положим \(a = \E\xi_{1} \in \R^{m}\). Тогда
	\[
		\frac{\xi_{1} + \dots + \xi_{n}}{n} \asto a.
	\]
\end{theorem}
\begin{proof}
	Автоматически следует из равносильности покомпонентной и общей сходимости и усиленного ЗБЧ для случайных величин.
\end{proof}

\subsection{Многомерное нормальное распределение}
Перед тем, как приступать к статистике, нужно ввести понятие многомерного нормального распределения. 
\begin{definition}
	Пусть есть случайный вектор \(\xi \in \R^{m}\). Будем говорить, что он подчиняется \emph{многомерному нормлаьному распределению}, если его характеристическая функция имеет вид
	\[
		\phi_{\xi}(\mathbf{t}) = \E{e^{i\left\langle\mathbf{t}, \xi\right\rangle}} = \exp\left\{i\left\langle \mathbf{a}, \mathbf{t}\right\rangle - \frac{1}{2}\left\langle \mathbf{\Sigma t}, \mathbf{t}\right\rangle\right\},
	\]
	где \(\mathbf{a} \in \R^{m}\)~--- некоторый фиксированный вектор, а \(\mathbf{\Sigma}\)~--- некоторая симметрическая и неотрицательно определённая матрица \(m \times m\). В таком случае пишут, что \(\xi \sim \mathcal{N}(\mathbf{a}, \mathbf{\Sigma})\).
\end{definition}

Есть одна весьма полезная теорема, которая даёт три эквивалентных определения.
\begin{theorem}[о трёх эквивалентных определениях]
	Пусть \(\xi = (\xi_{1}, \dots, \xi_{n})\)~--- случайный вектор. Тогда следующие утверждения равносильны:
	\begin{enumerate}
		\item Случайный вектор \(\xi\) имеет нормальное многомерное распределение.
		\item \(\xi = \mathbf{A}\eta + \mathbf{b}\) почти наверное, где \(\mathbf{A}\)~--- матрица размера \(n \times m\), \(\mathbf{b} \in \R^n\), а \(\eta = (\eta_{1}, \dots, \eta_{m})\), где \(\eta_{i}\) независимы в совокупности и имеют нормальное стандартное распределение.
		\item Для любого вектора \(\mathbf{x} \in \R^n\) случайная величина \(\left\langle\mathbf{x}, \xi\right\rangle\) имеет нормальное распределение.
	\end{enumerate}
\end{theorem}
\begin{proof}
	Докажем, что из \((2)\) следует \((3)\). Рассмотрим некоторый вектор \(\mathbf{x} \in \R^n\) и случайную величину \(\left\langle\mathbf{x}, \xi\right\rangle\):
	\begin{align*}
		\left\langle\mathbf{x}, \xi\right\rangle &= \left\langle\mathbf{x}, \mathbf{\Sigma}\eta + \mathbf{a}\right\rangle = \left\langle\mathbf{x}, \mathbf{a}\right\rangle + \left\langle\mathbf{x}, \mathbf{\Sigma}\eta\right\rangle \\
		&= \left\langle\mathbf{x}, \mathbf{a}\right\rangle + \sum_{k = 1}^{n} \left(\mathbf{\Sigma^{\top} x}\right)_{k}\eta_{k}
	\end{align*}
	Несложно понять, что это сумма нормальных случайных величин, что есть нормальная случайная величина.
	
	Теперь покажем, что из \((3)\) следует \((1)\). Пусть \(\left\langle\mathbf{x}, \xi\right\rangle \sim \mathcal{N}(a_{\mathbf{x}}, \sigma_{\mathbf{x}}^2)\). Тогда
	\begin{gather*}
		a_{\mathbf{x}} = \E{\left\langle\mathbf{x}, \xi\right\rangle} = \sum_{k = 1}^{n} \E{x_{k}\xi_{k}} = \sum_{k = 1}^{n} x_{k}\E{\xi_{k}} = \left\langle\mathbf{x}, \E{\xi}\right\rangle \\
		\sigma_{\mathbf{x}}^2 = \D{\left\langle\mathbf{x}, \xi\right\rangle} =  \D{\mathbf{x}^{\top}\xi} = \mathbf{x}^{\top}\D{\xi}\mathbf{x} = \left\langle\D{\xi}\mathbf{x}, \mathbf{x}\right\rangle
	\end{gather*}
	Осталось посчитать характеристическую функцию \(\xi\):
	\begin{align}
		\phi_{\xi}(\mathbf{x}) &= \E{e^{i\left\langle\mathbf{x}, \xi\right\rangle}} = \phi_{\left\langle\mathbf{x}, \xi\right\rangle}(1) = \exp\left\{ia_{\mathbf{x}} - \frac{1}{2}\sigma_{\mathbf{x}}^2\right\} \\
		&= \exp\left\{i\left\langle\E{\xi}, \mathbf{x}\right\rangle - \frac{1}{2}\left\langle\D{\xi}\mathbf{x}, \mathbf{x}\right\rangle\right\}.
	\end{align}
	А это и означает, что \(\xi\) имеет многомерное нормальное распределение.
	
	Теперь докажем, что из (1) следует (2). Так как \(\mathbf{\Sigma}\)~--- симметрическая и неотрицательно определённая матрица, то существует ортогональное преобразование \(\mathbf{C}\) такое, что
	\[
		\mathbf{C \Sigma C}^{\top} = \mathbf{D} = \operatorname{diag}(d_1, \dots, d_k, 0, \dots, 0),
	\]
	где \(d_1, \dots, d_k > 0\). Тогда рассмотрим случайную величину \(\xi' = \mathbf{C}(\xi - \mathbf{a})\). Найдём её характеристическую функцию:
	\begin{align*}
		\phi_{\xi'}(\mathbf{t}) &= \E{e^{i\left\langle \eta, \mathbf{t}\right\rangle}} = \E{e^{i\left\langle \mathbf{C}(\xi - \mathbf{a}), \mathbf{t}\right\rangle}} = \E{e^{i\left\langle (\xi - \mathbf{a}), \mathbf{C}^{\top}\mathbf{t}\right\rangle}} \\
		&= e^{-i\left\langle \mathbf{a}, \mathbf{C}^{\top}\mathbf{t}\right\rangle}\E{e^{i\left\langle \xi, \mathbf{C}^{\top}\mathbf{t}\right\rangle}}
	\end{align*}
	
	Заметим, что \(\E{e^{i\left\langle \xi, \mathbf{C}^{\top}\mathbf{t}\right\rangle}} = \phi_{\xi}(\mathbf{C^{\top}t})\). Тогда
	\begin{align*}
		e^{-i\left\langle \mathbf{a}, \mathbf{C}^{\top}\mathbf{t}\right\rangle}\E{e^{i\left\langle \xi, \mathbf{C}^{\top}\mathbf{t}\right\rangle}} &= \exp\left\{-i\left\langle \mathbf{a}, \mathbf{C}^{\top}\mathbf{t}\right\rangle + i\left\langle \mathbf{a}, \mathbf{C^{\top}t}\right\rangle - \frac{1}{2}\left\langle \mathbf{\Sigma}\mathbf{C^{\top}t}, \mathbf{C^{\top}t}\right\rangle\right\} \\
		&= \exp\left\{-\frac{1}{2}\left\langle \mathbf{C \Sigma C^{\top}t}, \mathbf{t}\right\rangle\right\} = \exp\left\{-\frac{1}{2}\left\langle \mathbf{Dt}, \mathbf{t}\right\rangle\right\} \\
		&= \exp\left\{-\frac{1}{2}\sum_{i = 1}^{k}d_i t_i^2\right\} = \prod_{i = 1}^{k} e^{-\frac{d_i t_i^2}{2}}
	\end{align*}
	
	Получаем, что
	\[
		\begin{cases}
			\xi'_i \sim \mathcal{N}(0, d_i),& i \in \{1, 2, \dots, k\} \\
			\xi'_i = 0 \text{ почти наверное},& i \in \{k + 1, \dots, n\}
		\end{cases}
	\]
	
	Введём случайные величины \(\eta_i = \frac{\xi'_i}{\sqrt{d_i}}\) для всех \(i \in \{1, 2, \dots, k\}\) и построим по ним случайный вектор \(\eta = (\eta_1, \dots, \eta_k)^{\top}\). Далее, построим следующую матрицу:
	\[
		\mathbf{B} = 
		\begin{pmatrix}
			\sqrt{d_1} & \dots &0 \\
			0 & \ddots & 0 \\
			0 & \dots & \sqrt{d_k} \\
			0 & \dots & 0 \\
			\vdots & \ddots & \vdots \\
			0 & \dots & 0
		\end{pmatrix}_{n \times k}
	\]
	
	Тогда несложно понять, что \(\mathbf{B}\eta = \xi'\) почти наверное. Отсюда получаем, что
	\[
		\xi = \mathbf{C^{\top}B}\eta + \mathbf{a}\qedhere
	\]
\end{proof}