\section{Лекция от 01.11.2016}

\begin{definition}
    Пусть функция $ F(x), x \in \R $, удовлетворяет свойствам из леммы, то есть: 
    \begin{enumerate}
        \item
        $F(x)$ неубывающая;
        \item
        \(\lim\limits_{x\to +\infty} F(x)= 1;\)
        \item
        \(\lim\limits_{x\to -\infty} F(x)= 0;\)
        \item
        $F(x)$ непрерывна справа.
    \end{enumerate}   
    Тогда такую функцию будем называть \emph{функцией распределения} на прямой.
\end{definition}

Следующая фундаментальная и очень важная теорема будет введена без доказательства.

\begin{theorem}[Теорема Каратеодори о продолжении меры]
    Пусть $ \Omega $ --- некоторое множество, $ \A $ -- алгебра подмножеств $ \Omega $. Пусть вероятностная мера \(\Pr_0: \A \mapsto [0; 1] \) удовлетворяет следующим свойствам:
    \begin{enumerate}
        \item
        \( \Pr_0(\Omega) = 1;  \)
        \item 
        \( \Pr_0 \) --- счетно-аддитивна на \(\A \).
    \end{enumerate}     
    Тогда существует (и притом единственна) вероятностная мера $ \Pr $ на $ \sigma(\A) $ такая, что мера $ \Pr $ является продолжением меры $ \Pr_0 $, иными словами \(\forall A \in \A \ \Pr_0(\A) = \Pr(\A)   \).
\end{theorem}

\begin{theorem}[Взаимно-однозначное соответствие функций распределения и мер на прямой]
Пусть \(F(x), x \in \R \) --- функция распределения на прямой. Тогда существует (и притом единственна) вероятностная мера $ \Pr $ на \(\mathcal{B} \), такая что $ F(x) $ --- её функция распределения, то есть \(\forall x \in \R \ \Pr\left((-\infty; x] \right) = F(x) \).
\end{theorem}

\begin{proof}
    Рассмотрим алгебру $ \A $, состоящую из конечных объединений непересекающихся полуинтервалов вида \((a; b],\ \forall A \in \A \) имеет вид 
    \[
    \A = \bigcup\limits_{k=1}^{n}(a_k; b_k],\  -\infty \leq a_1 < b_1 < a_2 < b_2 < \ldots < b_n < +\infty.
    \]    
    Зададим на $ \A $ меру $ \Pr_0 $ по правилу 
    \[
        \Pr_0(\A) = \sum\limits_{k = 1}^{n}\left( F(b_k) - F(a_k)\right).    
    \] 
    Тогда \(\Pr_0(\R) = F(+\infty) - F(-\infty) = 1 - 0 = 1 \) и по построению $ \Pr_0 $ будет конечно-аддитивной мерой.
    
    Для того чтобы воспользоваться теоремой Каратеодори, вероятностная мера $ \Pr_0 $ должна обладать свойством счетной аддитивности. Но вместо того, чтобы доказывать это свойство напрямую, воспользуемся \emph{теоремой о непрерывности вероятностной меры} из предыдущей лекции и докажем эквивалентное условие. Покажем, что мера $ \Pr_0 $ является непрерывной в нуле. \par
    
    Итак, нужно проверить, выполняется ли
    \[
    \lim\limits_{n \to \infty}\Pr_0(A_n) = 0
    \]
    для любых множеств \(A_1, A_2, \ldots \in \A \) таких, что \(A_{n + 1} \subset A_n,\ \bigcap\limits_{n = 1}^{\infty}A_n = \emptyset \)
    
    Заметим, что для любого полуинтервала $ (a, b] $ и для любого $ \delta > 0 $ можно взять такое $ a' > a $, что: 
    \[
        \Pr_0\left((a; b] \right) - \Pr_0\left((a'; b]\right) \leq \delta
    \]  
    \[
        \Pr_0\left((a; b] \right) - \Pr_0\left((a'; b]\right) = \left(F(b) - F(a)\right) - \left(F(b) - F(a') \right) = F(a') - F(a). 
    \]
    
    Зафиксируем произольное $ \epsilon > 0 $.
    В силу непрерывности справа такое значение $ a' $ можно подобрать.
    Тогда 
    \[
        \forall A_n \exists B_n \in \A, такое что \Pr_0(A_n) - \Pr_0(B_n) \leq \epsilon2^{-n},\ B_n \in A_n \textbf{ и } [B_n] \in A_n, \text { где } [B_n] \text { есть замыкание } B_n. 
    \]
    
    Пусть сначала все $ A_n $ лежат внутри $ [-N; N] $ для некоторого $ N \in \N $. Мы знаем, что \(\bigcap\limits_{n=1}^{\infty}A_n = \emptyset. \) Следовательно, \(\bigcap\limits_{n=1}^{\infty}[B_n] = \emptyset. \) 
    
    По принципу компактности существует $ n_0 $ такой, что из \(\bigcap\limits_{n=1}^{n_0}[B_n] = \emptyset \) следует \(\bigcap\limits_{n=1}^{n_0}B_n = \emptyset. \) Тогда:
    \[
        \Pr_0(A_{n_0}) = \Pr_0\left(A_{n_0} \setminus \bigcap\limits_{n=1}^{n_0}B_n\right)
    \]
    
    Следующий переход основан на том, что если \(\omega \in \A_n \setminus \bigcap\limits_{n=1}^{n_0}B_n \), то существует такой номер $ k $, что \(\omega \notin B_k\), а значит \(\omega \in \A_{n_0} \setminus B_k \).
    
    
    \begin{multline*} 
    \Pr_0\left(A_{n_0} \setminus \bigcap\limits_{n=1}^{n_0}B_n\right)
    \leq \sum\limits_{k=1}^{n_0}\Pr_0\left(A_{n_0} \setminus B_k \right) \leq \sum\limits_{k=1}^{n_0} \Pr_0\left(A_k \setminus B_k \right) \leq \\ \leq
    \sum\limits_{k=1}^{n_0} \left( \Pr_0\left(A_k\right) - \Pr_0\left(b_k\right) \right) \leq 
    \sum\limits_{k=1}^{n_0}\epsilon2^{-k} \leq \epsilon.
    \end{multline*}
    
    Получили, что \(\Pr_0(A_{n_0}) \leq \epsilon\). Значит, 
    \[
        \forall n > n_0 \ \Pr_0(A_n) \leq \epsilon \implies  \lim\limits_{n \to \infty}\Pr_0(A_n) = 0.
    \]
    
   Если в $ A_1 $ есть бесконечные полуинтервалы, то выберем полуинтервал $ (-N; N] $, что \(\Pr_0\left((-N; N] \right) \geq 1 - \frac{\epsilon}{2}\) и рассмотрим \(A_n' = A_n \cap (-N; N].  \)
   
   По доказанному выше \(\Pr_0(A_n') \leq \frac{\epsilon}{2} \) при $ n > n_0(\epsilon) $ и 
   \[
       \Pr_0(A_n) = \Pr_0\left(A_n' \cup \left(A_n \cup (-\infty; -N] \right) \cup \left(A_n \cap (N; +\infty)  \right)  \right) \leq \Pr_0\left(A_n'\right) + \Pr_0\left(\R \setminus (-N; N] \right) \leq \epsilon.
   \]
   
   По теореме Каратеодори искомая $ \Pr $ существует и единственна.

\end{proof}