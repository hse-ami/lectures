\section{Лекция 13 от 12.12.2016 \\ Ряды Тейлора}

\subsection{Дифференцирование степенных рядов}

В предыдущей лекции мы говорили о таком понятии, как степенные ряды. Продолжим.
\par Рассмотрим степенной ряд $\sum\limits_{n = 0}^{\infty}c_n (x-x_0)^n$, но для удобства сдвинем его центр, точку $x_0$, в нуль, получив тем самым ряд $\sum\limits_{n = 0}^{\infty}c_n x^n$. Рассмотрим к этому ряду другой ряд, составленный из производных исходного ряда (впредь будем его именовать <<новым>> рядом). Он будет иметь вид
\[
    \sum\limits_{n = 1}^{\infty}nc_n x^{n - 1} = \sum\limits_{n = 0}^{\infty}(n+1)c_{n + 1} x^n.
\]
\begin{Statement}
    Радиус сходимости нового ряда и исходного совпадают.
\end{Statement}
\begin{proof}
	Радиус сходимости нового ряда совпадает с радиусом сходимости ряда $\sum\limits_{n = 1}^{\infty}c_n n x^n$, так как мы просто умножаем на фиксированное число $x$. Тогда по формуле Коши-Адамара:
    \[ 
        R = \cfrac{1}{\overline{\lim\limits_{n\to\infty}}\sqrt[n]{|c_n|n}} = \cfrac{1}{\overline{\lim\limits_{n\to\infty}}\sqrt[n]{|c_n|}\underbrace{\lim\limits_{n\to\infty}\sqrt[n]{n}}_{\to 1}} = \cfrac{1}{\overline{\lim\limits_{n\to\infty}}\sqrt[n]{|c_n|}}.
    \]
    А это и есть исходный радиус.
\end{proof}
Выведем отсюда следствие, которое назовём теоремой.
\begin{Theorem}
    Внутри интервала сходимости сумма степенного ряда $\sum\limits_{n=0}^{\infty}c_nx_n$ дифференцируема и её производная равна $\sum\limits_{n=0}^{\infty}c_n n x^{n-1}$.
\end{Theorem}
\begin{proof}
    Возьмём произвольную точку $x$ из интервала сходимости. Найдём $\delta > 0$ такое, что $[x-\delta, x + \delta]$ лежит в интервале сходимости и используем теорему о почленном дифференцировании функциональных рядов.
\end{proof}
\begin{Consequence}
    Внутри интервала сходимости сумма степенного ряда бесконечно дифференцируема, и её $k$-я производная совпадает с суммой ряда из $k$-х производных.
\end{Consequence}
\begin{Consequence}
    Сумма ряда $\sum\limits_{n=0}^{\infty}c_n \cfrac{x^{n+1}}{n + 1}$ имеет тот же радиус сходимости, что и исходный ряд, и внутри интервала сходимости является первообразной суммы исходного ряда.
\end{Consequence}
Из равномерной сходимости степенного ряда на каждом отрезке множества сходимости можно вывести следующее:
\begin{Consequence}
    Пусть $[a,b]$ лежит в множестве сходимости степенного ряда $\sum\limits_{n = 0}^{\infty}c_n x^n$. Тогда 
    $$
    \int\limits_a^b\left(\sum\limits_{n = 0}^{\infty}c_n x^n\right)dx = \sum\limits_{n= 0}^{\infty} \int\limits_a^bc_n x^ndx = \sum\limits_{n=0}^\infty c_n \frac{b^{n+1} - a^{n+1}}{n+1}.
    $$
\end{Consequence}

\subsection{Функции, представимые как сумма степенного ряда}

Из следствия 1 сразу следует утверждение:
\begin{Statement}
    Пусть $I$ --- невырожденный промежуток. Если функция $f$ представима в виде суммы степенного ряда, то она бесконечно дифференцируема.
\end{Statement}

Покажем теперь, что функция не может представляться разными степенными рядами.  Действительно, будем поочерёдно дифференцировать левую и правую части нашего равенства функции и её степенного ряда (считаем, что радиус сходимости не нулевой):
    \begin{gather*}
        S(x) = \sum\limits_{n = 0}^{\infty} c_n x^n;\\
        S'(0) = c_1, \qquad S'(x) = \sseries c_n n x^{n-1};\\
        S''(0) = c_2\cdot 2!, \qquad S''(x) = \sum\limits_{n=2}^{\infty} c_n n(n-1) x^{n-2}; \\
        \ldots\\
        S^{(k)}(0) = c_k \cdot k!, \qquad S^{(k)}(x) = \sum\limits_{n=k}^{\infty} c_n \frac{n!}{(n-k)!} x^{n-k}, \\
        \ldots
    \end{gather*}

Отсюда сразу следует, что \textbf{если функция представима в виде степенного ряда на некотором множестве, то этот ряд совпадает с её рядом Тейлора}.

При этом не любая бесконечно дифференцируемая функция представима степенными рядом. Вспомним пример Коши --- бесконечно дифференцируемая функция, которая представима в ряд Тейлора лишь в точке 0:
\[
    f(x) = 
    \begin{cases}
        e^{-1/x^2}, & x \neq 0;\\
        0, & x = 0.
    \end{cases}
\]
\begin{Consequence}
    Если суммы степенных рядов $\sum\limits_{n=0}^{\infty}c_n x^n$ и $\sum\limits_{n=0}^{\infty}\widetilde{c_n} x^n$ совпадают в некоторой окрестности нуля, то эти ряды совпадают.
\end{Consequence}
\begin{proof}
    \[
        f(x) = \sum\limits_{n=0}^{\infty} c_n x^n = \sum\limits_{n=0}^{\infty} \widetilde{c_n} x^n.
    \]
    А в силу единственности разложения на невырожденном промежутке получим требуемое.
    $$
    c_n = \widetilde{c_n} = \dfrac{f^{(n)}(0)}{n!}.
    $$
\end{proof}
\begin{Comment}
    Совпадение в окрестности нуля тут можно заменить на совпадение в точках $x_n \neq 0$, $\lim\limits_{n\to \infty} x_n = 0$:
    \begin{gather*}
        c_0 = S(0) = \lim\limits_{n \to \infty} S(x_n) = \lim\limits_{n \to \infty} \widetilde{S}(x_n) = \widetilde{S}(0) = \widetilde{c_0}.
    \end{gather*}
    Теперь вычитаем $c_0$ и делим на $x$. Тогда равенство останется. И так далее.
\end{Comment}

\subsection{Представимость в виде ряда Тейлора}

\begin{Theorem}
    Пусть $I$ --- невырожденный промежуток и $f \in C^{\infty}(I),\; x_0 \in I$. Также пусть известно, что $\exists A, B>0,\; \forall n\in \N,\; \forall x\in I,\; |f^{(n)}(x)| \leq A\cdot B^n$. Тогда 
    \[
        \sum\limits_{n = 0}^{\infty} \cfrac{f^{(n)}(x_0)}{n!}(x-x_0)^n = f(x)
    \]
    на промежутке $I$.    Иными словами, функция представима своим рядом Тейлора.
\end{Theorem}
Перед доказательством заметим, что ряд $\sum\limits_{n=0}^{\infty}\cfrac{C^n}{n!}$ сходится по признаку Д'Аламбера для всякого положительного $C$, откуда получаем, что $\lim\limits_{n \to \infty}\cfrac{C^n}{n!} = 0$.
\begin{proof}
    Запишем разность частичной суммы ряда и значения функции, используя формулу Тейлора с остаточным членом в форме Лагранжа:
    \begin{gather*}
        \left| f(x) - \sum\limits_{n=0}^{N} \cfrac{f^{(n)}(x_0)}{n!} (x-x_0)^n\right| = 
        \left| \cfrac{f^{(N+1)}(\xi) (x-x_0)^{N+1}}{(N+1)!} \right| \leqslant \cfrac{AB^{N+1}|x-x_0|^{N+1}}{(N+1)!} \underset{n \to \infty}{\longrightarrow} 0.
    \end{gather*}
\end{proof}


Теперь рассмотрим, как получаются классические разложения в ряд Тейлора.
\begin{enumerate}
    \item $\sin(x)$ и $\cos(x)$. Ограничивая производные константой 1, получим требуемое:
    \begin{align*}
        &\sin(x) = \sum\limits_{n=0}^{\infty} \cfrac{(-1)^n x^{2n+1}}{(2n+1)!}\\
        &\cos(x) = \sum\limits_{n=0}^{\infty} \cfrac{(-1)^n x^{2n}}{(2n)!}
    \end{align*}
    \item $e^x$. Пусть $A>0$ --- произвольное число. Тогда на промежутке $(-A, A)$ ряд сходится, если мы применим ограничение производных как $e^A$.
    \item $\ln(1+x)$. <<Если делать в лоб, с ним всё грустно>>\copyright Лектор. Можно воспользоваться вспомогательным рядом:
    \[
        \cfrac{1}{x+1} = 1 - x + x^2 - \ldots\\
    \]
    который сходится на $(-1, 1)$ как геометрическая прогрессия, а затем почленно проинтегрировать, получив
    \[
        \ln(1+x) = x - \cfrac{x^2}{2} + \cfrac{x^3}{3} - \ldots.
    \]
    Из непрерывности равенство с интервала $(-1, 1)$ можно продолжить на полуотрезок $(-1, 1]$.
\end{enumerate}
